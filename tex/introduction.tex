\section{Introduction}%
\label{sec:introduction}

A moment analysis is a special case of a \emph{generalized Fourier
series}, \ie the expansion of a function into a complete set of
orthogonal functions.  Assume a set $\cBrk{\varphi_i(x)}$ of complex
base functions with $i \in \mathbb{N}^+$, which are square-integrable
over the interval~$R$ and fulfill the orthogonality relation
\begin{equation}
  \label{eq:orthogonality}
  \int_R \dif{x}\, \varphi_i(x)\, \varphi_j^*(x)
  = c_i\, \delta_{ij}.
\end{equation}
Here,
\begin{equation}
  \label{eq:normalization}
  c_i
  = \int_R \dif{x}\, \Abs{\varphi_i(x)}^2
\end{equation}
is the normalization of $\varphi_i(x)$.

If the set of orthogonal functions is complete, \ie if the functions
fulfill
\begin{equation}
  \label{eq:completeness}
  \sum_{i = 1}^\infty \frac{1}{c_i}\, \varphi_i(x)\, \varphi_i^*(x')
  = \delta(x' - x),
\end{equation}
we can decompose any arbitrary complex function $f(x)$ into a series
of these base functions, \ie
\begin{equation}
  \label{eq:gen_fourier_series}
  f(x)
  = \sum_{i = 1}^\infty a_i\, \varphi_i(x),
\end{equation}
where the expansion coefficients are given by\footnote{%
  \Cref{eq:expansion_coefficient} follows from orthogonality:
  \begin{equation*}
    \int_R \dif{x}\, f(x)\, \varphi_i^*(x)
    = \int_R \dif{x}\, \sum_{i = 1}^\infty a_j\, \varphi_j(x)\, \varphi_i^*(x)
    = \sum_{j = 1}^\infty a_j\, \int_R \dif{x}\, \varphi_j(x)\, \varphi_i^*(x)
    = \sum_{j = 1}^\infty a_j\, c_j\, \delta_{ji} = a_i\, c_i
  \end{equation*}
  and hence
  \begin{equation*}
    a_i
    = \frac{1}{c_i} \int_R \dif{x}\, f(x)\, \varphi_i^*(x).
  \end{equation*}
  We can prove \cref{eq:gen_fourier_series} by inserting
  \cref{eq:expansion_coefficient}, using the completeness in
  \cref{eq:completeness}:
  \begin{equation*}
    f(x)
    = \sum_{i = 1}^\infty \frac{1}{c_i} \int_R \dif{x'}\, f(x')\, \varphi_i^*(x')\, \varphi_i(x)
    = \int_R \dif{x'}\, f(x')\, \sum_{i = 1}^\infty \frac{1}{c_i} \varphi_i^*(x')\, \varphi_i(x)
    = \int_R \dif{x'}\, f(x')\, \delta(x' - x) = f(x).
  \end{equation*}%
}
\begin{equation}
  \label{eq:expansion_coefficient}
  a_i
  = \frac{1}{c_i} \int_R \dif{x}\, f(x)\, \varphi_i^*(x).
\end{equation}

This decomposition is unitary (Parseval's theorem), \ie
\begin{equation}
  \int_R \dif{x}\, f(x)\, f^*(x)
  \equalUsing{\cref{eq:gen_fourier_series}} \int_R \dif{x}\, \sum_{ij} a_i\, \varphi_i(x)\, a_j^*\, \varphi_j^*(x)
  = \sum_{ij} a_i\, a_j^*\, \dUnderbrace{\int_R \dif{x}\, \varphi_i(x)\, \varphi_j^*(x)}{\equalUsing{\cref{eq:orthogonality}} c_i\, \delta_{ij}}
  = \sum_i c_i \abs{a_i}^2.
\end{equation}

In this note, we will apply this decomposition technique to analyze
the angular distributions of two (pseudo)scalar particles produced in
inelastic high-energy scattering reactions.
