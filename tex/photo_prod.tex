\section{Photoproduction with linearly polarized beam}%
\label{sec:photoprod}

\subsection{Moment decomposition}%
\label{sec:photoprod:moment}

Systems of two distinguishable spinless particles can also be produced
in photo-production reactions using stationary proton targets.  As in
the case of diffractive reactions discussed in \cref{sec:diffraction},
we are interested in the intermediate meson resonances~$X$ that are
produced in these reactions and that decay into the observed
two-(pseudo)scalar system.  A typical example for such a process is
the reaction
\begin{equation}
  \gamma\, p \to X^0\, p \to \etaOrPr\pi^0\, p.
\end{equation}
How to analyze these reactions was studied in detail by the JPAC
collaboration in~\refsCite{Mathieu:2019fts,Mathieu:2019gxo}.  We will
follow these references here and will derive some of the
equations.\todo{Add figure with scattering process}

The analysis is based on the measured intensity distribution, \ie the
number density of events, which is decomposed into partial-wave
amplitudes:\footnote{\label{fn:photoprod_ampl_redef}This corresponds
to Eq.~(A3) in \refCite{Mathieu:2019fts} with the phase-space
factor~$\kappa$ absorbed into the partial-wave amplitudes, \ie with
the redefinition $\mathcal{T} \to \mathcal{T} / \sqrt{\kappa}$ (see
also footnote~6 in \refCite{Mathieu:2019fts}), and with Eq.~(A5)
inserted.}
\begin{equation}
  \label{eq:photoprod_intensity}
  \mathcal{I}(\Omega, \Phi; w, t)
  = \frac{\dif{N}}{\dif{w}\, \dif{t}\, \dif{\Omega}\, \dif{\Phi}}
  = \sum_{\substack{\ell m \\ \ell' m'}}^\infty
  Y_\ell^m(\Omega)~
  \dUnderbrace{\sum_{\substack{\lambda = \pm 1 \\ \mathclap{\lambda_1, \lambda_2 = \pm 1/2}}}
  \mathcal{T}^\ell_{\!m\, \lambda; \lambda_1\, \lambda_2}(w, t)\,
  \varrho^\gamma_{\lambda\, \lambda'}(\Phi)\,
  \mathcal{T}^{\ell' *}_{\!m'\, \lambda'; \lambda_1\, \lambda_2}(w, t)}{\equiv \varrho^{\ell\, \ell'}_{m\, m'}(w, t, \Phi)}~
  Y_{\ell'}^{m' *}(\Omega).
\end{equation}
Here, $N$~is the (acceptance-corrected) number of measured events,
$w$~is the invariant mass of the two-(pseudo)scalar system, $t$~is the
squared four-momentum transferred from the beam to the target
particle, $\Omega = (\theta, \phi)$ is the direction of one of the two
(pseudo)scalar mesons the~$X$ decays into, measured in the
Gottfried-Jackson or helicity rest frame of~$X$, and $\Phi$ is the
azimuthal\todo{correct?} angle between the polarization vector of the
linearly polarized photon and the reaction plane in the $X$~rest
frame.  The $\mathcal{T}^\ell_{\!m\, \lambda; \lambda_1\,
\lambda_2}(w, t)$ are the partial-wave amplitudes that correspond to
an intermediate state with a spin, which is given by the relative
orbital angular momentum~$\ell$ between the two-(pseudo)scalar mesons,
and a spin projection~$m$ \wrt the chosen quantization axis
(Gottfried-Jackson frame: beam direction; helicity frame: momentum
direction of~$X$).  This intermediate state is produced in the
interaction of a photon with helicity~$\lambda = \pm 1$ and a target
proton with helicity~$\lambda_1 = \pm 1/2$; the recoil proton has
helicity~$\lambda_2 = \pm 1/2$.  The angular distribution of the
$X$~decay products is given by the spherical harmonics
$Y_\ell^m(\Omega)$~\cite{wikipedia:sphericalHarm}; the dependence on
the polarization angle~$\Phi$ by the photon spin-density
matrix~$\mat{\upvarrho}^\gamma(\Phi)$.  In
\cref{eq:photoprod_intensity}, $\varrho^{\ell\, \ell'}_{m\, m'}(w, t,
\Phi)$ is the spin-density matrix element of~$X$ analogous to the
diffractive case in \cref{eq:diffraction_intensity_refl_spin-dens}.

Using the fact that the three Pauli matrices $\vec{\mat{\upsigma}} =
(\mat{\upsigma}_1, \mat{\upsigma}_2, \mat{\upsigma}_3)^T$ and
$\mat{I}_{2 \times 2}$ form a complete set in the space of Hermitian
$2 \times 2$ matrices, we can expand $\mat{\upvarrho}^\gamma(\Phi)$,
\ie
\begin{equation}
  \mat{\upvarrho}^\gamma(\Phi)
  = \frac{1}{2}\, \mat{I}_{2 \times 2} + \frac{1}{2}\, \vec{P}_\gamma(\Phi) \cdot \vec{\mat{\upsigma}}.
\end{equation}
Here, $\vec{P}_\gamma(\Phi)$ describes the photon polarization.  Its
length~$0 \leq P_\gamma \leq 1$ is the degree of polarization and its
direction depends on the kind of polarization:\footnote{See Eq.~(19)
in \refCite{Schilling:1969um}.}
\begin{equation}
  \vec{P}_\gamma(\Phi)
  = \begin{cases*}
    P_\gamma\, (0, 0, \lambda)^T                & for circularly polarized photons with $\lambda = \pm 1$, \\
    -P_\gamma\, (\cos 2 \Phi, \sin 2 \Phi, 0)^T & for linearly polarized photons with polarization angle~$\Phi$.
  \end{cases*}
\end{equation}
Hence, for linearly polarized photons, the intensity distribution in
\cref{eq:photoprod_intensity} can be written as the sum of three
terms:\footnote{See Eq.~(B4) in \refCite{Mathieu:2019fts}.}
\begin{equation}
  \label{eq:photoprod_intensity_sum}
  \mathcal{I}(\Omega, \Phi; w, t)
  = \mathcal{I}_0(\Omega; w, t)
  - \mathcal{I}_1(\Omega; w, t)\, P_\gamma\, \cos 2 \Phi
  - \mathcal{I}_2(\Omega; w, t)\, P_\gamma\, \sin 2 \Phi
\end{equation}
with the intensity components
\begin{equation}
  \label{eq:photoprod_intensity_components}
  \mathcal{I}_i(\Omega; w, t)
  = \sum_{\substack{\ell m \\ \ell' m'}}^\infty
  Y_\ell^m(\Omega)\,
  \prescript{i}{}{\varrho}^{\ell\, \ell'}_{m\, m'}(w, t)\,
  Y_{\ell'}^{m' *}(\Omega)
\end{equation}
and the spin-density matrix components\footnote{See Eq.~(11) in
\refCite{Mathieu:2019fts}.}
\begin{align}
  \label{eq:photoprod_rho_0}
  \prescript{0}{}{\varrho}^{\ell\, \ell'}_{m\, m'}(w, t)
  ={}& \frac{1}{2}\quad \sum_{\substack{\lambda = \pm 1 \\ \mathclap{\lambda_1, \lambda_2 = \pm 1/2}}}
  \mathcal{T}^\ell_{\!m\, \lambda; \lambda_1\, \lambda_2}(w, t)\,
  \mathcal{T}^{\ell' *}_{\!m'\, \lambda; \lambda_1\, \lambda_2}(w, t)
  \\
  \label{eq:photoprod_rho_1}
  \prescript{1}{}{\varrho}^{\ell\, \ell'}_{m\, m'}(w, t)
  ={}& \frac{1}{2}\quad \sum_{\substack{\lambda = \pm 1 \\ \mathclap{\lambda_1, \lambda_2 = \pm 1/2}}}
  \mathcal{T}^\ell_{\!m\, {-\lambda}; \lambda_1\, \lambda_2}(w, t)\,
  \mathcal{T}^{\ell' *}_{\!m'\, \lambda; \lambda_1\, \lambda_2}(w, t)
  \\
  \label{eq:photoprod_rho_2}
  \prescript{2}{}{\varrho}^{\ell\, \ell'}_{m\, m'}(w, t)
  ={}& \frac{\imag}{2}\quad \sum_{\substack{\lambda = \pm 1 \\ \mathclap{\lambda_1, \lambda_2 = \pm 1/2}}}
  \lambda\,
  \mathcal{T}^\ell_{\!m\, {-\lambda}; \lambda_1\, \lambda_2}(w, t)\,
  \mathcal{T}^{\ell' *}_{\!m'\, \lambda; \lambda_1\, \lambda_2}(w, t).
\end{align}
Note that neither the $\mathcal{I}_i$ nor the
$\prescript{i}{}{\varrho}^{\ell\, \ell'}_{m\, m'}$ depend on~$\Phi$.
Furthermore, the spin-density matrix elements of~$X$ are given by
\begin{equation}
  \label{eq:photoprod_spin_dens_sum}
  \varrho^{\ell\, \ell'}_{m\, m'}(w, t, \Phi)
  = \prescript{0}{}{\varrho}^{\ell\, \ell'}_{m\, m'}(w, t)
  - \prescript{1}{}{\varrho}^{\ell\, \ell'}_{m\, m'}(w, t)\, P_\gamma\, \cos 2 \Phi
  - \prescript{2}{}{\varrho}^{\ell\, \ell'}_{m\, m'}(w, t)\, P_\gamma\, \sin 2 \Phi.
\end{equation}

In \cref{eq:photoprod_intensity_sum}, the three intensity components
$\mathcal{I}_i(\Omega; w, t)$ (as well as the three spin-density
matrix components in \cref{eq:photoprod_spin_dens_sum}) are modulated
by different $\Phi$~dependences.  $\mathcal{I}_0(\Omega; w, t)$ is the
unpolarized intensity and hence constant in~$\Phi$.  It is equivalent
to the intensity for the diffractive case in
\cref{eq:diffraction_intensity_refl_spin-dens}.  The other two
components $\mathcal{I}_{1, 2}(\Omega; w, t)$ depend on the photon
polarization and are hence modulated by different $\Phi$~dependences.
Note that the functions $\cBrk[1]{f_0(\Phi), f_1(\Phi), f_2(\Phi)} =
\cBrk[1]{1, \cos 2 \Phi, \sin 2 \Phi}$ that modulate the intensity
components constitute an orthogonal set of functions, \ie
\begin{equation}
  \label{eq:photoprod_orthogonality_phi}
  \int_{-\pi}^{+\pi}\!\!\! \dif{\Phi}\, f_i(\Phi)\, f_j(\Phi)
  = (1 + \delta_{i 0}) \pi\, \delta_{i j}.
\end{equation}

Like for the diffractive case, the partial-wave analysis as well as
the moment decomposition are performed in narrow kinematic cells in
the $(w, t)$ plane, assuming that within each cell all quantities are
in good approximation independent of~$w$ and~$t$.  To simplify
notation we hence omit the~$w$ and $t$~dependencies in all formulas
below, \ie these formulas are valid in a given $(w, t)$ cell.

We decompose the intensity in \cref{eq:photoprod_intensity_sum} into
spherical harmonics to obtain the moments, analogously to the
diffractive case in \cref{eq:diffraction_intensity_moments}.  Due to
the orthogonality of the $\mathcal{I}_i$ terms in $\Phi$~space, we can
decompose each intensity component separately:\footnote{See Eq.~(A8)
in \refCite{Mathieu:2019fts}.}
\begin{align}
  \label{eq:photoprod_intensity_moments_norm_unpol}
  \mathcal{I}_0(\Omega)
  ={}& \sum_{L M}^\infty \sqrt{\frac{2 L + 1}{4 \pi}}\, H_0(L, M)\, Y_L^M(\Omega)
  \\
  \label{eq:photoprod_intensity_moments_norm_pol}
  \mathcal{I}_{1, 2}(\Omega)
  ={}& -\sum_{L M}^\infty \sqrt{\frac{2 L + 1}{4 \pi}}\, H_{1, 2}(L, M)\, Y_L^M(\Omega).
\end{align}
Here, we have used the same normalization as in
\cref{sec:diffraction:moments_norm}.
\Cref{eq:photoprod_intensity_moments_norm_unpol} is equivalent to
\cref{eq:diffraction_intensity_moments_norm}.  The minus sign in
\cref{eq:photoprod_intensity_moments_norm_pol} is introduced in order
to compensate that the corresponding intensity components contribute
negatively to the intensity distribution in
\cref{eq:photoprod_intensity_sum}.\footnote{Equivalently, the minus
sign ensures that $H_1(0, 0) \geq 0$ for positive reflectivity waves
(see Appendix~D in \refCite{Mathieu:2019fts}).}

The corresponding moments are
\begin{align}
  \label{eq:photoprod_moments_norm_unpol}
  H_0(L, M)
  ={}& \frac{1}{2 \pi}\, \sqrt{\frac{4 \pi}{2 L + 1}}\, \int_{4 \pi}\!\!\! \dif{\Omega}\, \int_{-\pi}^{+\pi}\!\!\! \dif{\Phi}\,
  \mathcal{I}(\Omega, \Phi)\, Y_L^{M *}(\Omega)
  \\
  \label{eq:photoprod_moments_norm_pol}
  H_{1, 2}(L, M)
  ={}& \frac{1}{P_\gamma}\, \frac{1}{\pi}\, \sqrt{\frac{4 \pi}{2 L + 1}}\, \int_{4 \pi}\!\!\! \dif{\Omega}\, \int_{-\pi}^{+\pi}\!\!\! \dif{\Phi}\,
  \mathcal{I}(\Omega, \Phi)\, Y_L^{M *}(\Omega) \times \begin{cases*}
    \cos 2 \Phi & for $i = 1$, \\
    \sin 2 \Phi & for $i = 2$.
  \end{cases*}
\end{align}
Here, $H_0(L, M)$ becomes identical to
\cref{eq:diffraction_moments_norm} for the diffractive case if the
intensity distribution is independent of~$\Phi$.  The two polarized
moments $H_{1, 2}(L, M)$ represent orthogonal angular distributions
in~$\Phi$ (see \cref{eq:photoprod_orthogonality_phi}).


\subsection{Reflectivity basis}%
\label{sec:photoprod:reflectivity}

In Appendix~D of \refCite{Mathieu:2019fts}, the reflectivity basis is
introduced by defining the partial-wave amplitudes\footnote{See
Eq.~(D1) in \refCite{Mathieu:2019fts}.}\footnote{Alternative
approaches to apply the reflectivity basis are discussed in
\refCite{Salgado:2020}.}
\begin{equation}
  \label{eq:photoprod_amplitude_refl}
  \prescript{\refl}{}{\mathcal{T}}^\ell_{\!m; \lambda_1\, \lambda_2}
  \equiv \frac{1}{2}\, \sBrk{\mathcal{T}^\ell_{\!m\, {+1}; \lambda_1\, \lambda_2}
  - \refl\, (-1)^m\, \mathcal{T}^\ell_{\!{-m}\, {-1}; \lambda_1\, \lambda_2}(w, t)}.
\end{equation}
The partial-wave amplitudes in the reflectivity basis are hence linear
combinations of the partial-wave amplitudes with opposite photon
helicities and opposite $X$~spin projection quantum numbers.  In
\cref{eq:photoprod_amplitude_refl} the reflectivity quantum number
$\refl = \pm 1$ effectively replaces the photon helicity $\lambda =
\pm 1$.

Formulating the intensity model in the reflectivity basis has the
advantage that in the high-energy limit at leading order the
reflectivity defined in \cref{eq:photoprod_amplitude_refl} corresponds
to the naturality of the particle exchanged in the scattering process
(see Appendices~C and~D in \refCite{Mathieu:2019fts}).  Another
advantage of the reflectivity basis is that parity directlys relates
the partial-wave amplitudes with opposite helicities of the target and
the recoil proton, \ie\footnote{See Eq.~(D3) in
\refCite{Mathieu:2019fts}.}
\begin{equation}
  \prescript{\refl}{}{\mathcal{T}}^\ell_{\!m; {-\lambda_1}\, {-\lambda_2}}
  = \refl\, (-1)^{\lambda_1 - \lambda_2}\, \prescript{\refl}{}{\mathcal{T}}^\ell_{\!m; \lambda_1\, \lambda_2}.
\end{equation}
This means for given~\refl, $\ell$, and~$m$, only two of the four
possible partial-wave amplitudes are independent, namely the proton
spin-flip amplitude\footnote{See Eq.~(D4) in
\refCite{Mathieu:2019fts}.}
\begin{align}
  \prescript{\refl}{}{\mathcal{T}}^\ell_{\!m; {+1}\, {-1}}
  \equiv{}& \sBrk{\ell}^{(\refl)}_{m; 0}
  \intertext{and the proton spin-non-flip amplitude}
  \prescript{\refl}{}{\mathcal{T}}^\ell_{\!m; {+1}\, {+1}}
  \equiv{}& \sBrk{\ell}^{(\refl)}_{m; 1}.
\end{align}
Furthermore, since parity is conserved the spin-flip and spin-non-flip
amplitudes do not interfere.  Introducing these parity constraints in
\crefrange{eq:photoprod_intensity_components}{eq:photoprod_rho_2}
would be more difficult because in the conventional basis, parity
relates amplitudes that in addition have opposite~$m$ and~$\lambda$
quantum numbers.  However, in the reflectivity basis\footnote{See
Eq.~(D7) in \refCite{Mathieu:2019fts}.}
\begin{equation}
  \label{eq:photoprod_rho_refl}
  \prescript{i}{}{\varrho}^{\ell\, \ell'}_{m\, m'}
  = \prescript{i}{}{\varrho}^{(+)\, \ell\, \ell'}_{m\, m'} + \prescript{i}{}{\varrho}^{(-)\, \ell\, \ell'}_{m\, m'}
\end{equation}
with\footnote{See Eq.~(D8) in \refCite{Mathieu:2019fts} and
\cref{fn:photoprod_ampl_redef}.}
\begin{align}
  \label{eq:photoprod_rho_0_refl}
  \prescript{0}{}{\varrho}^{(\refl)\, \ell\, \ell'}_{m\, m'}
  ={}& \sum_{k = 0, 1} \rBrk[2]{\sBrk{\ell}^{(\refl)}_{m; k}\, \sBrk{\ell'}^{(\refl) *}_{m'; k}
  + (-1)^{m - m'}\, \sBrk{\ell}^{(\refl)}_{{-m}; k}\, \sBrk{\ell'}^{(\refl) *}_{{-m'}; k}}
  \\
  \label{eq:photoprod_rho_1_refl}
  \prescript{1}{}{\varrho}^{(\refl)\, \ell\, \ell'}_{m\, m'}
  ={}& -\refl \sum_{k = 0, 1} \rBrk[2]{(1)^m\, \sBrk{\ell}^{(\refl)}_{{-m}; k}\, \sBrk{\ell'}^{(\refl) *}_{m'; k}
  + (-1)^{m'}\, \sBrk{\ell}^{(\refl)}_{m; k}\, \sBrk{\ell'}^{(\refl) *}_{{-m'}; k}}
  \\
  \label{eq:photoprod_rho_2_refl}
  \prescript{2}{}{\varrho}^{(\refl)\, \ell\, \ell'}_{m\, m'}
  ={}& -\imag\, \refl \sum_{k = 0, 1} \rBrk[2]{(1)^m\, \sBrk{\ell}^{(\refl)}_{{-m}; k}\, \sBrk{\ell'}^{(\refl) *}_{m'; k}
  - (-1)^{m'}\, \sBrk{\ell}^{(\refl)}_{m; k}\, \sBrk{\ell'}^{(\refl) *}_{{-m'}; k}}.
\end{align}
In \cref{eq:photoprod_rho_refl}, we have used the fact that
partial-wave amplitudes with opposite~\refl do not
interfere.\footnote{See Eq.~(D5) in \refCite{Mathieu:2019fts}.}



Since the spin of the target and the
recoil proton is not measured, the two





\subsection{Relation between moments and partial-wave amplitudes}%
\label{sec:photoprod:moments_pw}

As in the diffractive case, we see that the moments are linear
combinations of the corresponding components of the spin-density
matrix of~$X$ by inserting \cref{eq:spherical_harm_prod} into
\cref{eq:photoprod_intensity_components} and comparing with
\cref{eq:photoprod_intensity_moments_norm_unpol,eq:photoprod_intensity_moments_norm_pol},
\ie\footnote{See Eq.~(A9) in \refCite{Mathieu:2019fts}.}
\begin{align}
  H_0(L, M)
  ={}& \sum_{\substack{\ell m \\ \ell' m'}}^\infty \sqrt{\frac{2 \ell' + 1}{2 \ell + 1}}
  \clebsch{\ell'}{0}{L}{0}{\ell}{0}\, \clebsch{\ell'}{m'}{L}{M}{\ell}{m}\,
  \prescript{0}{}{\varrho}^{\ell\, \ell'}_{m\, m'}
  \\
  H_{1, 2}(L, M)
  ={}& -\sum_{\substack{\ell m \\ \ell' m'}}^\infty \sqrt{\frac{2 \ell' + 1}{2 \ell + 1}}
  \clebsch{\ell'}{0}{L}{0}{\ell}{0}\, \clebsch{\ell'}{m'}{L}{M}{\ell}{m}\,
  \prescript{1, 2}{}{\varrho}^{\ell\, \ell'}_{m\, m'}.
\end{align}
Here, the Clebsch-Gordan coefficients restrict the sum to those
quantum numbers, for which $\ell' + L + \ell = \text{even}$ (see
\cref{eq:ang_mom_sum}), $\abs{\ell' - L} \leq \ell \leq \ell' + L$,
and $m = m' + M$.  Conversely, this means that a partial-wave
amplitude with orbital angular momentum~$\ell$ contributes to all
moments~$H_i$ with~$L$ from~0 up to $2 \ell$.
