\section{Diffractive scattering}%
\label{sec:diffraction}

\subsection{Moment decomposition}%
\label{sec:diffraction:moment}

We start with the simplest reaction, which is the production of a
system of two distinguishable spinless particles in the strong
interaction of a highly energetic beam particle (usually pion or kaon)
with a stationary nucleon or nuclear target.  We are interested in the
short-lived intermediate states~$X$ (resonances) that are produced in
these reactions and that decay into the observed two-(pseudo)scalar
system.  A typical example for such a process is the reaction
\begin{equation}
  \pi^-\, p \to X^-\, p \to \etaOrPrPim\, p.
\end{equation}
How to analyze these reactions was studied in detail by S.U.~Chung in
\refCite{Chung:1997qd}.  We will follow this reference here and will
derive some equations.\todo{Add figure with scattering process}

The analysis is based on the measured intensity distribution, \ie the
number density of events, which is decomposed into partial-wave
amplitudes:
\begin{equation}
  \label{eq:intensity_diffraction}
  \mathcal{I}(\Omega; w, t)
  = \frac{\dif{N}}{\dif{w}\, \dif{t}\, \dif{\Omega}}
  = \Abs[3]{\sum_{\ell m}^\infty \mathcal{T\!}_{\ell\, m}(w, t)\, Y_\ell^m(\Omega)}^2.
\end{equation}
Here, $N$~is the (acceptance-corrected) number of measured events,
$w$~is the invariant mass of the two-(pseudo)scalar system, $t$~is the
squared four-momentum~$t$ transferred from the beam to the target
particle, and $\Omega = (\theta, \phi)$ is the direction of one of the
two (pseudo)scalar mesons the~$X$ decays into, measured in the
Gottfried-Jackson rest frame of~$X$.  The $\mathcal{T\!}_{\ell\, m}(w, t)$
are the partial-wave amplitudes that correspond to an intermediate
state with a spin given by the relative orbital angular
momentum~$\ell$ between the two-(pseudo)scalar mesons and a spin
projection~$m$ \wrt the beam axis.  The angular distribution of the
$X$~decay products is given by the spherical harmonics
$Y_\ell^m(\Omega)$~\cite{wikipedia:sphericalHarm}.  In
\cref{eq:intensity_diffraction}, we have assumed that the effect of
the target spin is negligible and that all partial wave amplitudes are
fully coherent.

To measure the $\mathcal{T\!}_{\ell\, m}$, which contain the information
about the produced resonances $X$, the intensity model in
\cref{eq:intensity_diffraction} is fit to the measured $\Omega$
distribution in narrow $(w, t)$ cells.  Within each kinematic cell,
$\mathcal{T\!}_{\ell\, m}$ is treated as constant.  To simplify notation,
we hence will omit the~$m$ and $t$~dependencies in all formulas below.

In addition to the partial-wave analysis, which decomposes the
\emph{amplitudes} into spherical harmonics, we can exploit the fact
that the $Y_\ell^m(\Omega)$ constitute a complete orthonormal set of
base functions on the surface of a unit sphere, \ie they fullfil the
completeness relation (see Eq.~(1) in Sec.~5.6.1 of
\refCite{Varshalovich:1988krb})
\begin{equation}
  \label{eq:spherical_harm_complete}
  \sum_{\ell m}^\infty Y_\ell^m(\Omega)\, Y_\ell^{m *}(\Omega')
  = \delta(\phi' - \phi)\, \delta(\theta' - \theta)
\end{equation}
and the orthonormality relation\footnote{%
Throughout the text, we use
\begin{equation}
  \int_{-1}^{+1}\!\!\! \dif{\cos\theta} \int_{-\pi}^{+\pi}\!\!\! \dif{\phi}
  = \int_{4 \pi}\!\!\! \dif{\Omega}.
\end{equation}
} (see Eq.~(2) in Sec.~5.6.1 of \refCite{Varshalovich:1988krb})
\begin{equation}
  \label{eq:spherical_harm_orthonorm}
  \int_{4 \pi}\!\!\! \dif{\Omega}\, Y_\ell^m(\Omega)\, Y_{\ell'}^{m' *}(\Omega)
  = \delta_{\ell \ell'}\, \delta_{m m'},
\end{equation}
and instead decompose the \emph{intensity} into spherical harmonics.

Using \cref{eq:gen_fourier_series,eq:expansion_coefficient} we obtain:
\begin{equation}
  \label{eq:intensity_moments_diffraction}
  \mathcal{I}(\Omega)
  = \sum_{L M}^\infty H(L, M)\, Y_L^M(\Omega),
\end{equation}
with the moments\footnote{The expansion into a series of spherical
harmonics is also referred to as Laplace
series~\cite{MathWorld:LaplaceSeries}.}
\begin{equation}
  \label{eq:moments_diffraction}
  H(L, M)
  = \int_{4 \pi}\!\!\! \dif{\Omega}\, \mathcal{I}(\Omega)\, Y_L^{M *}(\Omega).
\end{equation}
It is important to note that although
\cref{eq:intensity_moments_diffraction} and
\cref{eq:intensity_diffraction} use the same basis functions, the two
represent different decompositions.  In
\cref{eq:intensity_diffraction} the amplitudes are decomposed into
spherical harmonics, \ie partial-wave amplitudes, whereas in
\cref{eq:intensity_moments_diffraction} the intensity is decomposed.
Whereas the former decomposition is based on a physics model of the
reaction, the moment decomposition is purely mathematical.  An
advantage of the moment decomposition is that it is unique, whereas
the partial-wave decomposition may have ambiguities (see \eg
\refCite{Chung:1997qd}).  Another advantage of the moment
decomposition is that the moments can be rather easily calculated from
the (acceptance-corrected) data.  The downside of this approach is
that the physics interpretation of the moments is difficult because
they are only indirectly related to the partial-wave amplitudes.  This
will be discussed in more detail in \cref{sec:diffraction:moments_pw}.


\subsection{Reflectivity basis}%
\label{sec:diffraction:reflectivity}

It is often advantageous to express \cref{eq:intensity_diffraction} in
the reflectivity basis
\begin{equation}
  \label{eq:refl_def_diffraction}
  \ket{\refl, \ell, m}
  \equiv \mathcal{N}_m\, \sBrk[2]{\ket{\ell, m} - \refl\, {(-1)}^m\, \ket{\ell, {-m}}},
\end{equation}
where the spin states $\ket{\refl, \ell, m}$ of~$X$ are eigenstates of
the reflection operator through the production plane that is spanned
by the beam particle and~$X$.  The corresponding eigenvalues~$\refl =
\pm 1$, the \emph{reflectivities}, are chosen such that they
correspond to the naturality of the exchange particle in the
scattering process.  The normalization factor
\begin{equation}
  \label{eq:refl_norm}
  \mathcal{N}_m
  = \begin{cases*}
      1 / \sqrt{2} & for $m > 0$, \\
      1 / 2        & for $m = 0$, \\
      0            & for $m < 0$
    \end{cases*}
\end{equation}
ensures that the multiplicity of $2 \ell + 1$ of the spin states is
conserved by constraining~$m$ to be non-negative.  It is important to
note that the definition in \cref{eq:refl_def_diffraction} forbids
positive-reflectivity states with $m = 0$.

The reflectivity basis is implemented by introducing the functions
\begin{equation}
  \label{eq:spherical_harm_refl_diffraction}
  \prescript{\refl}{}{Y}_\ell^m(\Omega)
  \equiv \mathcal{N}_m\, \sBrk[2]{Y_\ell^m(\Omega) - \refl\, {(-1)}^m\, Y_\ell^{(-m)}(\Omega)}.
\end{equation}
Note that with
\begin{equation}
  \label{eq:spherical_harm_sym}
  Y_\ell^{m *}(\Omega)
  = {(-1)}^m\, Y_\ell^{(-m)}(\Omega)
\end{equation}
(see Eq.~(1) in Sec.~5.4 of \refCite{Varshalovich:1988krb}) and the
definition of the spherical harmonics (see Eq.~(1) in Sec.~5.2 of
\refCite{Varshalovich:1988krb}),
\begin{equation}
  \label{eq:sspherical_harm_def}
  Y_\ell^m(\Omega)
  = \dUnderbrace{\sqrt{\frac{(2 \ell + 1)}{4 \pi}\, \frac{(\ell -m)!}{(\ell + m)!}}\, P_\ell^m(\cos\theta)}{= Y_\ell^m(\theta, 0)}\, e^{\imag\, m\, \phi},
\end{equation}
it follows
from \cref{eq:spherical_harm_refl_diffraction} that
\begin{equation}
  \prescript{(+)}{}{Y}_\ell^m(\Omega)
  = 2 \imag\, \mathcal{N}_m\, Y_\ell^m(\theta, 0)\, \sin(m\, \phi)
  \quad\text{and}\quad
  \prescript{(-)}{}{Y}_\ell^m(\Omega)
  = 2 \mathcal{N}_m\, Y_\ell^m(\theta, 0)\, \cos(m\, \phi),
\end{equation}
\ie $\prescript{(+)}{}{Y}_\ell^m(\Omega)$ is purely imaginary and
$\prescript{(-)}{}{Y}_\ell^m(\Omega)$ purely real.

For convenience, we introduce
\begin{equation}
  \label{eq:small_y_def}
  Y_\ell^m(\theta, 0)
  \equiv y_\ell^m(\theta)
  \quad\text{so that}\quad
  Y_\ell^m(\theta, \phi)
  = y_\ell^m(\theta)\, e^{\imag\, m\, \phi}.
\end{equation}
Note that $y_\ell^m(\theta)$ is real-valued.

Replacing $Y_\ell^m(\Omega)$ in \cref{eq:intensity_diffraction} by
\cref{eq:spherical_harm_refl_diffraction} and using the fact that
amplitudes with opposite reflectivities do not interfere, we obtain
\begin{equation}
  \label{eq:intensity_diffraction_refl}
  \mathcal{I}(\Omega)
  = \sum_{\refl = \pm 1} \Abs[3]{\sum_{\ell m}^\infty \prescript{\refl}{}{\mathcal{T\!}}_{\ell\, m}\, \prescript{\refl}{}{Y}_\ell^m(\Omega)}^2.
\end{equation}

Expanding the absolute-value-squared term in
\cref{eq:intensity_diffraction_refl}, we get
\begin{align}
  \mathcal{I}(\Omega)
  ={}& \sum_{\refl = \pm 1} \sum_{\substack{\ell m \\ \ell' m'}}^\infty
  \prescript{\refl}{}{\mathcal{T\!}}_{\ell\, m}\, \prescript{\refl}{}{\mathcal{T\!}}_{\ell'\, m'}^*\,
  \prescript{\refl}{}{Y}_\ell^m(\Omega)\, \prescript{\refl}{}{Y}_{\ell'}^{m' *}(\Omega)
  \\
  \label{eq:intensity_diffraction_refl_spin-dens}
  ={}& \sum_{\refl = \pm 1} \sum_{\substack{\ell m \\ \ell' m'}}^\infty
  \prescript{\refl}{}{\varrho}^{\ell\, \ell'}_{m\, m'}\,
  \prescript{\refl}{}{Y}_\ell^m(\Omega)\, \prescript{\refl}{}{Y}_{\ell'}^{m' *}(\Omega),
\end{align}
where we have introduced the spin-density matrix of~$X$,
\begin{equation}
  \label{eq:diffraction_refl_spin-dens}
  \prescript{\refl}{}{\varrho}^{\ell\, \ell'}_{m\, m'}
  = \prescript{\refl}{}{\mathcal{T\!}}_{\ell\, m}\, \prescript{\refl}{}{\mathcal{T\!}}_{\ell'\, m'}^*.
\end{equation}


\subsection{Relation between moments and partial-wave amplitudes}%
\label{sec:diffraction:moments_pw}

In order to relate the partial-wave
amplitudes~$\prescript{\refl}{}{\mathcal{T\!}}_{\ell\, m}$ and the
moments~$H(L, M)$ we insert \cref{eq:spherical_harm_refl_diffraction}
into \cref{eq:intensity_diffraction_refl_spin-dens} and use the
relation (see Eq.~(A.15) in \refCite{Chung:1971ri})
\begin{equation}
  Y_{\ell'}^{m' *}(\Omega)\, Y_{\ell}^m(\Omega)
  = \sum_{L M}^\infty \sqrt{\frac{2 L + 1}{4 \pi}}\, \sqrt{\frac{2 \ell' + 1}{2 \ell + 1}}\, \clebsch{\ell'}{0}{L}{0}{\ell}{0}\, \clebsch{\ell'}{m'}{L}{M}{\ell}{m}\, Y_L^M(\Omega).
\end{equation}
Here, \clebsch{j_1}{m_1}{j_2}{m_2}{J}{M} are the Clebsch-Gordan
coefficients for the coupling of the spin states $\ket{j_1\; m_1}$
and~$\ket{j_2\; m_2}$ to $\ket{J\; M}$.  Note that the Clebsch-Gordan
coefficient \clebsch{\ell'}{0}{L}{0}{\ell}{0} requires that
\begin{equation}
  \label{eq:ang_mom_sum}
  \ell' + L + \ell
  = \text{even}
\end{equation}
(see Eq.~(32) in Sec.~8.5.2 of \refCite{Varshalovich:1988krb}).  Using
the above, we obtain
\begin{align}
  \mathcal{I}(\Omega)
  ={}&
    \sum_{\refl = \pm 1} \sum_{\substack{\ell m \\ \ell' m'}}^\infty
    \prescript{\refl}{}{\varrho}^{\ell\, \ell'}_{m\, m'}\,
    \mathcal{N}_m\, \mathcal{N}_{m'}
    \sBrk[2]{Y_\ell^m(\Omega) - \refl\, {(-1)}^m\, Y_\ell^{(-m)}(\Omega)}
    \sBrk[2]{Y_{\ell'}^{m' *}(\Omega) - \refl\, {(-1)}^{m'}\, Y_{\ell'}^{(-m') *}(\Omega)}
  \\
  ={}& \begin{multlined}[t][0.85\columnwidth]
    \sum_{\refl = \pm 1} \sum_{\substack{\ell m \\ \ell' m'}}^\infty
    \prescript{\refl}{}{\varrho}^{\ell\, \ell'}_{m\, m'}\,
    \sum_{L M}^\infty \sqrt{\frac{2 L + 1}{4 \pi}}\, \sqrt{\frac{2 \ell' + 1}{2 \ell + 1}}\, \clebsch{\ell'}{0}{L}{0}{\ell}{0} \\
    \shoveleft{\times \mathcal{N}_m\, \mathcal{N}_{m'}\, \Big[ \clebsch{\ell'}{m'}{L}{M}{\ell}{m}                       - \refl\, {(-1)}^{m'}\, \clebsch{\ell'}{-m'}{L}{M}{\ell}{m}} \\
      - \refl\, {(-1)}^m\, \clebsch{\ell'}{m'}{L}{M}{\ell}{-m} + {(-1)}^{m + m'}\, \clebsch{\ell'}{-m'}{L}{M}{\ell}{-m} \Big]\,
    Y_L^M(\Omega).
  \end{multlined}
\end{align}
We rewrite the last Clebsch-Gordan coefficient in the square bracket
using the symmetry relation (see Eq.~(11) in Sec.~8.4.3 of
\refCite{Varshalovich:1988krb})
\begin{equation}
  \clebsch{\ell'}{-m'}{L}{M}{\ell}{-m}
  = \cdUnderbrace{{(-1)}^{\ell' + L - \ell}}{= \dUnderbrace{{(-1)}^{\ell' + L + \ell}}{= 1}\, \dUnderbrace{{(-1)}^{-2 \ell}}{= 1} = 1}\, \clebsch{\ell'}{m'}{L}{-M}{\ell}{m},
\end{equation}
where we have used \cref{eq:ang_mom_sum}.  Furthermore, ${(-1)}^{m + m'}
= {(-1)}^{2m}\, {(-1)}^{m' - m} = {(-1)}^M$ because the Clebsch-Gordan
coefficient enforces that $m' - M = m$.  With this we arrive at
\begin{multline}
  \mathcal{I}(\Omega)
  = \sum_{L M}^\infty \sum_{\refl = \pm 1} \sum_{\substack{\ell m \\ \ell' m'}}^\infty
    \sqrt{\frac{2 L + 1}{4 \pi}}\, \sqrt{\frac{2 \ell' + 1}{2 \ell + 1}}\,
    \prescript{\refl}{}{\varrho}^{\ell\, \ell'}_{m\, m'}\, \clebsch{\ell'}{0}{L}{0}{\ell}{0} \\
    \times \mathcal{N}_m\, \mathcal{N}_{m'}\, \Big[
      \clebsch{\ell'}{m'}{L}{M}{\ell}{m}
      + {(-1)}^M\, \clebsch{\ell'}{m'}{L}{-M}{\ell}{m} \\
      - \refl\, {(-1)}^{m'}\, \clebsch{\ell'}{-m'}{L}{M}{\ell}{m}
      - \refl\, {(-1)}^m\, \clebsch{\ell'}{m'}{L}{M}{\ell}{-m} \Big]\,
    Y_L^M(\Omega).
\end{multline}
Comparing with \cref{eq:intensity_moments_diffraction} we see that
\begin{multline}
  \label{eq:moments_pw_diffraction}
  H(L, M)
  = \sum_{\refl = \pm 1} \sum_{\substack{\ell m \\ \ell' m'}}^\infty
    \sqrt{\frac{2 L + 1}{4 \pi}}\, \sqrt{\frac{2 \ell' + 1}{2 \ell + 1}}\,
    \prescript{\refl}{}{\varrho}^{\ell\, \ell'}_{m\, m'}\, \clebsch{\ell'}{0}{L}{0}{\ell}{0} \\
    \times \mathcal{N}_m\, \mathcal{N}_{m'}\, \Big[
      \clebsch{\ell'}{m'}{L}{M}{\ell}{m}
      + {(-1)}^M\, \clebsch{\ell'}{m'}{L}{-M}{\ell}{m} \\
      - \refl\, {(-1)}^{m'}\, \clebsch{\ell'}{-m'}{L}{M}{\ell}{m}
      - \refl\, {(-1)}^m\, \clebsch{\ell'}{m'}{L}{M}{\ell}{-m} \Big].
\end{multline}

It is important to note that we can always calculate all moments from
a given set of partial-wave amplitudes using
\cref{eq:moments_pw_diffraction}.  However, in general the converse is
not true because there are unphysical sets of moments that do not
correspond to a partial-wave decomposition and even for a physical set
of moments we need to solve a system of quadratic equations in the
partial-wave amplitudes given by \cref{eq:moments_pw_diffraction,},
which may contain ambiguities.

The moments are linear combinations of the spin-density matrix
elements, \ie of intensities and interference terms of the
partial-wave amplitudes $\prescript{\refl}{}{\mathcal{T\!}}_{\ell\,
m}$.  \Cref{eq:moments_pw_diffraction} also shows that~$L$ and~$M$ are
only indirectly linked to the physical angular momentum quantum
numbers of the two-body system.  For given~$L$, the Clebsch-Gordan
coefficients limit the sum to those partial waves, for which $\ell' +
L + \ell$ is even and $\abs{\ell' - L} \leq \ell \leq \ell' + L$.
Conversely, this means that a partial-wave amplitude with orbital
angular momentum~$\ell$ contributes to moments from $L = 0$ up to $L =
2 \ell$.  It is important to note that each moment is an incoherent
sum of contributions from both reflectivities, \ie the moments do not
separate these contributions.

We can derive \cref{eq:moments_pw_diffraction} also by inserting
\cref{eq:intensity_diffraction_refl} into
\cref{eq:moments_diffraction} and using
\begin{equation}
  \label{eq:spherical_harm_clebsch}
  \int_{4 \pi}\!\!\! \dif{\Omega}\,
  Y_{\ell'}^{m' *}(\Omega)\, Y_L^{M *}(\Omega)\, Y_{\ell}^m(\Omega)
  = \sqrt{\frac{2 \ell' + 1}{4 \pi}}\, \sqrt{\frac{2 L + 1}{2 \ell + 1}}\, \clebsch{\ell'}{0}{L}{0}{\ell}{0}\, \clebsch{\ell'}{m'}{L}{M}{\ell}{m}
\end{equation}
(complex conjugate of Eq.~(4) in Sec.~5.9.1 of
\refCite{Varshalovich:1988krb}).  Doing so, we obtain
\begin{multline}
  \label{eq:moments_pw_diffraction_refl}
  H(L, M)
  = \sum_{\refl = \pm 1} \sum_{\substack{\ell m \\ \ell' m'}}^\infty
  \prescript{\refl}{}{\varrho}^{\ell\, \ell'}_{m\, m'}\,
  \mathcal{N}_m\, \mathcal{N}_{m'} \\
  \times \int_{4 \pi}\!\!\! \dif{\Omega}\,
  \sBrk[2]{Y_\ell^m(\Omega) - \refl\, {(-1)}^m\, Y_\ell^{(-m)}(\Omega)}
  \sBrk[2]{Y_{\ell'}^{m' *}(\Omega) - \refl\, {(-1)}^{m'}\, Y_{\ell'}^{(-m') *}(\Omega)}\,
  Y_L^{M *}(\Omega),
\end{multline}
which is equivalent to \cref{eq:moments_pw_diffraction}.


\subsection{Normalization of moments}%
\label{sec:diffraction:moments_norm}

In order to derive a more practical normalization, it is instructive
to look at the lowest moment $H(0, 0)$, which is a special case.  With
\cref{eq:moments_pw_diffraction} and
\begin{equation}
  \clebsch{\ell'}{m'}{0}{0}{\ell}{m}
  = \delta_{\ell \ell'}\, \delta_{m m'}
\end{equation}
we get
\begin{align}
  H(0, 0)
  ={}& \begin{multlined}[t]
    \sum_{\refl = \pm 1} \sum_{\substack{\ell m \\ \ell' m'}}^\infty
    \frac{1}{\sqrt{4 \pi}}\, \sqrt{\frac{2 \ell' + 1}{2 \ell + 1}}\,
      \prescript{\refl}{}{\varrho}^{\ell\, \ell'}_{m\, m'}\, \delta_{\ell \ell'} \\
      \times \rdUnderbrace{\mathcal{N}_m\, \mathcal{N}_{m'}\, \sBrk[2]{%
        \delta_{\ell \ell'}\, \delta_{m m'}
        + \delta_{\ell \ell'}\, \delta_{m m'}
        - \refl\, {(-1)}^{m'}\, \delta_{\ell \ell'}\, \delta_{m (-m')}
        - \refl\, {(-1)}^m\, \delta_{\ell \ell'}\, \delta_{(-m) m'}}}%
        {= \delta_{\ell \ell'}\, \delta_{m m'} \times
        \begin{cases*}
          \frac{1}{2}\, (1 + 1)                 & for $m > 0$, \\
          \frac{1}{4}\, (1 + 1 - \refl - \refl) & for $m = 0$
        \end{cases*}}
  \end{multlined} \notag
  \\
  \label{eq:moment_00_pw_diffraction}
  ={}& \frac{1}{\sqrt{4 \pi}} \sum_{\refl = \pm 1} \sum_\ell^\infty \sum_{m = 0}^\ell \prescript{\refl}{}{\varrho}^{\ell\, \ell}_{m\, m},
\end{align}
where we have used \cref{eq:refl_norm} and $m, m' \geq 0$.  According
to the above, $H(0, 0)$ is the sum of all partial-wave
intensities, except for a factor $1 / \sqrt{4 \pi}$.

In order to make $H(0, 0)$ identical to the sum of all partial-wave
intensities, it is customary to change the normalization of the
moments using the replacement
\begin{equation}
  H(L, M)
  \to \sqrt{\frac{4 \pi}{2 L + 1}}\, H(L, M).
\end{equation}
Applying the above,
\cref{eq:intensity_moments_diffraction,eq:moments_diffraction} become
\begin{equation}
  \label{eq:intensity_moments_diffraction_norm}
  \mathcal{I}(\Omega)
  = \sum_{L M}^\infty \sqrt{\frac{2 L + 1}{4 \pi}}\, H(L, M)\, Y_L^M(\Omega),
\end{equation}
and
\begin{equation}
  \label{eq:moments_diffraction_norm}
  H(L, M)
  = \sqrt{\frac{4 \pi}{2 L + 1}}\, \int_{4 \pi}\!\!\! \dif{\Omega}\, \mathcal{I}(\Omega)\, Y_L^{M *}(\Omega).
\end{equation}
Using the relation between the spherical harmonics and the Wigner
$D$-functions~\cite{wikipedia:wignerD}:
\begin{equation}
  \sqrt{\frac{2 L + 1}{4 \pi}}\, D^{L *}_{M 0}(\phi, \theta, 0)
  = Y_L^M(\Omega),
\end{equation}
\cref{eq:intensity_moments_diffraction_norm,eq:moments_diffraction_norm}
become identical to Eqs.~(6) and~(9) in \refCite{Chung:1997qd}

Similarly, \cref{eq:moments_pw_diffraction} becomes
\begin{multline}
  \label{eq:moments_pw_diffraction_norm}
  H(L, M)
  = \sum_{\refl = \pm 1} \sum_{\substack{\ell m \\ \ell' m'}}^\infty
    \sqrt{\frac{2 \ell' + 1}{2 \ell + 1}}\,
    \prescript{\refl}{}{\varrho}^{\ell\, \ell'}_{m\, m'}\, \clebsch{\ell'}{0}{L}{0}{\ell}{0} \\
    \times \mathcal{N}_m\, \mathcal{N}_{m'}\, \Big[
      \clebsch{\ell'}{m'}{L}{M}{\ell}{m}
      + {(-1)}^M\, \clebsch{\ell'}{m'}{L}{-M}{\ell}{m} \\
      - \refl\, {(-1)}^{m'}\, \clebsch{\ell'}{-m'}{L}{M}{\ell}{m}
      - \refl\, {(-1)}^m\, \clebsch{\ell'}{m'}{L}{M}{\ell}{-m} \Big],
\end{multline}
which is equivalent to Eqs.~(29) and~(31) in \refCite{Chung:1997qd}.

For the remainder of this section we will use the unnormalized moments
defined in
\cref{eq:intensity_moments_diffraction_norm,eq:moments_diffraction_norm,eq:moments_pw_diffraction_norm}.


\subsection{Symmetry properties of moments}%
\label{sec:diffraction:moments_sym}

Since the moments are linear combinations of spin-density matrix
elements, the moments inherit their symmetry properties.  By
definition (\confer \cref{eq:diffraction_refl_spin-dens}), the
spin-density matrix is Hermitian, \ie
\begin{equation}
  \rBrk{\prescript{\refl}{}{\varrho}^{\ell\, \ell'}_{m\, m'}}^*
  = \prescript{\refl}{}{\varrho}^{\ell'\, \ell}_{m'\, m}.
\end{equation}
Using this together with \cref{eq:moments_pw_diffraction_norm} and
\begin{equation}
  \label{eq:clebsch_sym}
  \clebsch{\ell'}{m'}{L}{M}{\ell}{m}
  = \cdUnderbrace{{(-1)}^{\ell' + L - \ell}}{= 1~\text{because of \cref{eq:ang_mom_sum}}}\, \clebsch{L}{M}{\ell'}{m'}{\ell}{m}
  = {(-1)}^{L - M}\, \clebsch{\ell}{m}{L}{-M}{\ell'}{m'}
\end{equation}
(see Eq.~(10) in Sec.~8.4.3 of \refCite{Varshalovich:1988krb}), we get
\begin{align}
  H^*(L, M)
  ={}& \begin{multlined}[t][0.8\textwidth]
    \sum_{\refl = \pm 1} \sum_{\substack{\ell m \\ \ell' m'}}^\infty
    \sqrt{\frac{2 \ell' + 1}{2 \ell + 1}}\,
    \prescript{\refl}{}{\varrho}^{\ell'\, \ell}_{m'\, m}\, {(-1)}^L\, \clebsch{\ell}{0}{L}{0}{\ell'}{0} \\
    \shoveleft{\times \mathcal{N}_m\, \mathcal{N}_{m'}\, {(-1)}^L\, \Big[
      {(-1)}^{-M}\, \clebsch{\ell}{m}{L}{-M}{\ell'}{m'}
      + {(-1)}^{2M}\, \clebsch{\ell}{m}{L}{M}{\ell'}{m'}} \\
      - \refl\, {(-1)}^{-M}\, {(-1)}^{m'}\, \clebsch{\ell}{m}{L}{-M}{\ell'}{-m'}
      - \refl\, {(-1)}^{-M}\, {(-1)}^m\, \clebsch{\ell}{-m}{L}{-M}{\ell'}{m'} \Big]
  \end{multlined} \notag
  \\
  ={}& \begin{multlined}[t][0.8\textwidth]
    \dUnderbrace{{(-1)}^{2 L - M}}{= {(-1)}^M}\, \sum_{\refl = \pm 1} \sum_{\substack{\ell m \\ \ell' m'}}^\infty
    \sqrt{\frac{2 \ell' + 1}{2 \ell + 1}}\,
    \prescript{\refl}{}{\varrho}^{\ell'\, \ell}_{m'\, m}\, \clebsch{\ell}{0}{L}{0}{\ell'}{0} \\
    \shoveleft{\times \mathcal{N}_m\, \mathcal{N}_{m'}\, \Big[
      \clebsch{\ell}{m}{L}{-M}{\ell'}{m'}
      + {(-1)}^{-M}\, \clebsch{\ell}{m}{L}{M}{\ell'}{m'}} \\
      - \refl\, {(-1)}^{m'}\, \clebsch{\ell}{m}{L}{-M}{\ell'}{-m'}
      - \refl\, {(-1)}^m\, \clebsch{\ell}{-m}{L}{-M}{\ell'}{m'} \Big]
  \end{multlined} \notag
  \\
  \label{eq:diffraction_moment_sym_1}
  ={}& (-1)^M\, H(L, -M),
\end{align}
where in the last step we first made the replacements $\ell
\leftrightarrow \ell'$ and $m \leftrightarrow m'$ and then compared to
\cref{eq:moments_pw_diffraction_norm}.\footnote{%
  Alternatively, we can derive \cref{eq:diffraction_moment_sym_1} from
  the definition of the moments in \cref{eq:moments_diffraction} and
  using \cref{eq:spherical_harm_sym}:
  \begin{equation}
    H^*(L, M)
    = \int_{4 \pi}\!\!\! \dif{\Omega}\, \mathcal{I}(\Omega)\, Y_L^M(\Omega)
    = (-1)^M \int_{4 \pi}\!\!\! \dif{\Omega}\, \mathcal{I}(\Omega)\, Y_L^{(-M) *}(\Omega)
    = (-1)^M\, H(L, -M).
  \end{equation}
}  The above relation ensures that the intensity function in
\cref{eq:intensity_moments_diffraction_norm} is real-valued:
\begin{align}
  \mathcal{I}(\Omega)
  ={}& \sum_{L = 0}^\infty \sum_{M = -L}^{+L} \sqrt{\frac{2 L + 1}{4 \pi}}\, H(L, M)\, Y_L^M(\Omega) \notag
  \\
  ={}& \sum_{L = 0}^\infty \sqrt{\frac{2 L + 1}{4 \pi}} \sBrk[4]{\rdUnderbrace{\sum_{M = -L}^{-1} H(L, M)\, Y_L^M(\Omega)}%
    {= \sum_{M = +1}^{+L} H(L, -M)\, Y_L^{(-M)}(\Omega)
    \equalUsingTwo{\cref{eq:spherical_harm_sym}}{\cref{eq:diffraction_moment_sym_1}} {(-1)}^M\, H^*(L, M)\, {(-1)}^M\, Y_L^{M *}(\Omega)}
    + H(L, 0)\, Y_L^0(\Omega) + \sum_{M = +1}^{+L} H(L, M)\, Y_L^M(\Omega)} \notag
  \\
  ={}& \sum_{L = 0}^\infty \sqrt{\frac{2 L + 1}{4 \pi}} \sBrk[4]{H(L, 0)\, Y_L^0(\Omega) + \sum_{M = +1}^{+L}
  \rdUnderbrace{\cBrk{H(L, M)\,Y_L^M(\Omega) + H^*(L, M)\,Y_L^{M *}(\Omega)}}{= 2 \Re\!\sBrk{H(L, M)\, Y_L^M(\Omega)}}} \notag
  \\
  \label{eq:intensity_moments_diffraction_general}
  ={}& \sum_{L = 0}^\infty \sqrt{\frac{2 L + 1}{4 \pi}} \sum_{M = 0}^{L} (2 - \delta_{M 0})\, \Re\!\sBrk{H(L, M)\, Y_L^M(\Omega)}.
\end{align}
For the last step, we have used that $Y_L^0(\Omega)$ and hence also
$H(L, 0)$ are real-valued by construction.  Due to the above, we only
have to calculate the moments with $M \geq 0$.

Using \cref{eq:moments_pw_diffraction_norm} and the symmetry
properties of the Clebsch-Gordan coefficients in
\cref{eq:clebsch_sym}, we can show that
\begin{align}
  H(L, -M)
  ={}& \begin{multlined}[t][0.8\textwidth]
    \sum_{\refl = \pm 1} \sum_{\substack{\ell m \\ \ell' m'}}^\infty
    \sqrt{\frac{2 \ell' + 1}{2 \ell + 1}}\,
    \prescript{\refl}{}{\varrho}^{\ell\, \ell'}_{m\, m'}\, {(-1)}^L\, \clebsch{\ell}{0}{L}{0}{\ell'}{0} \\
    \times \mathcal{N}_m\, \mathcal{N}_{m'}\, {(-1)}^L\, \Big[
      {(-1)}^M\, \clebsch{\ell}{m}{L}{M}{\ell'}{m'}
      + {(-1)}^{-M}\, {(-1)}^M\, \clebsch{\ell}{m}{L}{-M}{\ell'}{m'} \\
      - {(-1)}^M\, \refl\, {(-1)}^{m'}\, \clebsch{\ell}{m}{L}{M}{\ell'}{-m'}
      - {(-1)}^M\, \refl\, {(-1)}^m\, \clebsch{\ell}{-m}{L}{M}{\ell'}{m'} \Big]
  \end{multlined} \notag
  \\
  ={}& \begin{multlined}[t][0.8\textwidth]
    \dUnderbrace{{(-1)}^{2 L - M}}{= {(-1)}^M}\, \sum_{\refl = \pm 1} \sum_{\substack{\ell m \\ \ell' m'}}^\infty
    \sqrt{\frac{2 \ell' + 1}{2 \ell + 1}}\,
    \prescript{\refl}{}{\varrho}^{\ell\, \ell'}_{m\, m'}\, \clebsch{\ell}{0}{L}{0}{\ell'}{0} \\
    \times \mathcal{N}_m\, \mathcal{N}_{m'}\, \Big[
      \clebsch{\ell}{m}{L}{M}{\ell'}{m'}
      + {(-1)}^{M}\, \clebsch{\ell}{m}{L}{-M}{\ell'}{m'} \\
      - \refl\, {(-1)}^{m'}\, \clebsch{\ell}{m}{L}{M}{\ell'}{-m'}
      - \refl\, {(-1)}^m\, \clebsch{\ell}{-m}{L}{M}{\ell'}{m'} \Big]
  \end{multlined} \notag
  \\
  \label{eq:diffraction_moment_sym_2}
  ={}& {(-1)}^M\, H(L, M),
\end{align}
where in the last step we first made the replacements $\ell
\leftrightarrow \ell'$ and $m \leftrightarrow m'$ and then compared to
\cref{eq:moments_pw_diffraction_norm}.

From \cref{eq:diffraction_moment_sym_1,eq:diffraction_moment_sym_2} follows that
\begin{equation}
  H^*(L, M) = H(L, M),
\end{equation}
\ie all moments must be real-valued.  Inserting
\Cref{eq:moments_diffraction_norm}, we obtain
\begin{align}
  \sqrt{\frac{4 \pi}{2 L + 1}}\, \int_{4 \pi}\!\!\! \dif{\Omega}\, \mathcal{I}(\Omega)\, Y_L^{M *}(\Omega)
  \mustBeEq{}&
  \sqrt{\frac{4 \pi}{2 L + 1}}\, \int_{4 \pi}\!\!\! \dif{\Omega}\, \mathcal{I}(\Omega)\, Y_L^M(\Omega) \\
  \sqrt{\frac{4 \pi}{2 L + 1}}\, \int_{4 \pi}\!\!\! \dif{\Omega}\, \mathcal{I}(\Omega)\, y_L^M(\theta)\, \sBrk{\cos(M \phi) + \imag \sin(M \phi)}
  \mustBeEq{}&
  \sqrt{\frac{4 \pi}{2 L + 1}}\, \int_{4 \pi}\!\!\! \dif{\Omega}\, \mathcal{I}(\Omega)\, y_L^M(\theta)\, \sBrk{\cos(M \phi) - \imag \sin(M \phi)}.
\end{align}
Consequently,
\begin{align}
  \label{eq:moments_diffraction_imag}
  0
  ={}& \sqrt{\frac{4 \pi}{2 L + 1}}\, \int_{4 \pi}\!\!\! \dif{\Omega}\, \mathcal{I}(\Omega)\, y_L^M(\theta)\, \sin(M\, \phi) \\
  \intertext{and}
  \label{eq:moments_diffraction_real}
  H(L, M)
  ={}& \sqrt{\frac{4 \pi}{2 L + 1}}\, \int_{4 \pi}\!\!\! \dif{\Omega}\, \mathcal{I}(\Omega)\, y_L^M(\theta)\, \cos(M\, \phi).
\end{align}

We can also rewrite \cref{eq:intensity_moments_diffraction_general}:
\begin{align}
  \mathcal{I}(\Omega)
  ={}& \sum_{L = 0}^\infty \sqrt{\frac{2 L + 1}{4 \pi}} \sum_{M = 0}^{L} (2 - \delta_{M 0})\, H(L, M)\, \Re\!\sBrk{Y_L^M(\Omega)} \notag
  \\
  \label{eq:intensity_moments_diffraction_real}
  ={}& \sum_{L = 0}^\infty \sqrt{\frac{2 L + 1}{4 \pi}} \sum_{M = 0}^{L} (2 - \delta_{M 0})\, H(L, M)\, y_L^M(\theta)\, \cos(M\, \phi).
\end{align}
This is identical to Eq.~(13) in \refCite{Chung:1997qd}.


\subsection{Calculation of moments from data}%
\label{sec:diffraction:moments_data}

\subsubsection{Ideal detector with perfect acceptance}%
\label{sec:diffraction:moments_data_no_acc}

Using \cref{eq:moments_diffraction_real}, we can calculate the
real-valued moments $H(L, M)$ with $M \geq 0$ for a given intensity
distribution~$\mathcal{I}(\Omega)$.  The moments with $M < 0$ are
linearly dependent and are given by
\cref{eq:diffraction_moment_sym_2}.  Assuming an ideal detector with
perfect acceptance, the measured events will follow the distribution
$\mathcal{I}(\Omega)$.  For a data sample with $N$~events, the best
estimates $\hat{H}(L, M)$ for the moments are hence given by replacing
the integral over $\mathcal{I}(\Omega)$ in
\cref{eq:moments_diffraction_real} by a sum over the measured events,
\ie
\begin{equation}
  \label{eq:moments_diffraction_estimate}
  \hat{H}(L, M)
  = \sum_{i = 1}^N \sqrt{\frac{4 \pi}{2 L + 1}}\, y_L^M(\theta_i)\, \cos(M\, \phi_i)
  \equiv \sum_{i = 1}^N f_{L M}(\Omega_i),
\end{equation}
where~$\theta_i$ and~$\phi_i$ are the angles measured for event~$i$.
Here, we have omitted the $4 \pi / N$ factor that one would normally
write when estimating an integral over the surface of the unit sphere.
In the literature, this is referred to as \emph{unnormalized
moments}\footnote{This is somewhat of a misnomer because also the
unnormalized moments have a well-defined normalization.  However, the
normalization is not absolute and hence these moments depend on the
number of events.} and with the chosen normalization (see
\cref{sec:diffraction:moments_norm}) it leads to
\begin{equation}
  \label{eq:diffraction_weight_00}
  H(0, 0)
  = N
  \quad\text{because}\quad
  f_{00}(\Omega)
  = 1
  = \text{const}.
\end{equation}
\todo{relate to \cref{eq:moment_00_pw_diffraction}? Why no
interference effect? integral of interference terms always 0?}This
means that unnormalized moments are expressed in units of events.

To estimate the uncertainty of $\hat{H}(L, M)$, we exploit that the
events are statistically independent and identically distributed
according to $\mathcal{I}(\Omega)$.\footnote{%
  For a data sample $\cBrk{(x_1, y_1), \ldots, (x_N, y_N)}$ of the
  random variables~$x$ and~$y$ with $N$~events, the covariance of the
  sums of these samples is
  \begin{equation}
    \cov[4]{\sum_{i = 1}^N x_i, \sum_{i = 1}^N y_i}
    = \exptVal[4]{\sum_{i j}^N x_i\, y_j} - \exptVal[4]{\sum_i^N x_i}\, \exptVal[4]{\sum_i^N y_i}
    = \sum_{i j}^N \rBrk[2]{\exptVal[1]{x_i\, y_j} - \exptVal[1]{x_i}\, \exptVal[1]{y_i}}
    = \sum_{i j}^N \cov[1]{x_i, y_j}.
  \end{equation}
  If the $N$~events are statistically independent then $\cov[1]{x_i,
  y_j} = 0$ for $i \neq j$.  Hence,
  \begin{equation}
    \cov[4]{\sum_{i = 1}^N x_i, \sum_{i = 1}^N y_i}
    = \sum_i^N \cov[1]{x_i, y_i}
    \equalUsing{\text{iid}} N\, \cov[1]{x, y},
  \end{equation}
  where we used the fact that the samples are independent and
  identically distributed (iid) random variables.
}  Therefore, the variance is given by
\begin{equation}
  \label{eq:moments_diffraction_var}
  \var[1]{\hat{H}(L, M)}
  = \sigma_{\hat{H}(L, M)}^2
  = \var[4]{\sum_{i = 1}^N f_{L M}(\Omega_i)}
  = \sum_{i = 1}^N \var[2]{f_{L M}(\Omega)}
  = N\, \var[1]{f_{L M}(\Omega)}.
\end{equation}
An unbiased estimate of the variance of $f_{L M}(\Omega)$ is the
sample variance, \ie
\begin{equation}
  \hat{V}\!\sBrk[1]{f_{L M}(\Omega)}
  = \frac{1}{N - 1}\, \sum_{i = 1}^N \sBrk[2]{f_{L M}(\Omega_i) - \mean{f_{L M}(\Omega)}}^2
\end{equation}
with the sample mean
\begin{equation}
  \mean{f_{L M}(\Omega)}
  = \frac{1}{N}\, \sum_{i = 1}^N f_{L M}(\Omega_i).
\end{equation}
Hence,
\begin{equation}
  \label{eq:moments_diffraction_stdDev}
  \hat{\sigma}_{\hat{H}(L, M)}^2
  = \frac{N}{N - 1}\, \sum_{i = 1}^N \sBrk[2]{f_{L M}(\Omega_i) - \mean{f_{L M}(\Omega)}}^2.
\end{equation}
This is analogous to Monte Carlo integration (see \eg\
\refCite{wikipedia:MonteCarloIntegration}).  Note that with
\cref{eq:diffraction_weight_00} we get
\begin{equation}
  \hat{\sigma}_{\hat{H}(0, 0)}
  = 0.
\end{equation}

Similarly, we can estimate the covariance
matrix~$\hat{\covMatSym}_{\!\hat{H}}$ for a given set of moment
estimates using the sample covariance:
\begin{align}
  (\hat{\covMatSym}_{\!\hat{H}})_{L M, L' M'}
  ={}& \cov[2]{\hat{H}(L, M), \hat{H}(L', M')}
  = N\, \cov[2]{f_{L M}(\Omega), f_{L' M'}(\Omega)} \notag
  \\
  \label{eq:diffraction_sample_cov}
  ={}& \frac{N}{N - 1}\, \sum_{i = 1}^N \sBrk[2]{f_{L M}(\Omega_i) - \mean{f_{L M}(\Omega)}}\, \sBrk[2]{f_{L' M'}(\Omega_i) - \mean{f_{L' M'}(\Omega)}}.
\end{align}
Note that from a numerical standpoint the above form is not the most
advantageous way to compute the covariance.  See
\refCite{wikipedia:CovarianceAlgorithm} for more suitable numerical
algorithms.


\subsubsection{Taking into account detection efficiency}%
\label{sec:diffraction:acceptance_corr}

The approach presented in \cref{sec:diffraction:moments_data_no_acc}
only works if detector effects can be neglected.  In reality, the
experimental data will be distorted by detection efficiency and
detector resolution.  Here, we assume that the resolution in the
angular variables is high enough so that we can safely neglect
resolution effects.  Hence, the measured intensity distribution is
given by
\begin{equation}
  \label{eq:diffraction_int_meas}
  \mathcal{I}_\text{meas}(\Omega)
  = \eta(\Omega)\, \mathcal{I}(\Omega),
\end{equation}
where $\eta(\Omega)$ is the detection efficiency for the reaction
under study\footnote{The detection efficiency includes all effects
that influence the probability to detect a given event, in particular
geometric acceptance of the detectors and inefficiencies introduced by
the detectors, by the event reconstruction, and by the trigger and
offline event selection.  In principle, in addition to~$\Omega$ the
detection efficiency depends on additional kinematic variables.  In
\cref{eq:diffraction_int_meas}, we have marginalized~$\eta$ over these
variables.  In order to perform this marginalization we need a
realistic model of the reaction that reproduces the kinematic
distributions in the variables we marginalize over sufficiently well.}
and $\mathcal{I}(\Omega)$ is the physical intensity distribution.  In case
resolution effects in~$\Omega$ are non-negligible for the channel
under study one has to apply the support-vector moment approach
developed by R.~Jones~\cite{Jones:2023}, which allows to unfold the
detector resolution.

Decomposing the measured intensity distribution into spherical
harmonics analogous to \cref{eq:intensity_moments_diffraction_general}
we obtain
\begin{equation}
  \mathcal{I}_\text{meas}(\Omega)
  = \sum_{L = 0}^\infty \sqrt{\frac{2 L + 1}{4 \pi}} \sum_{M = 0}^{L} (2 - \delta_{M 0})\, \Re\!\sBrk{H_\text{meas}(L, M)\, Y_L^M(\Omega)}
\end{equation}
with the measured
moments\footnote{\label{fn:complex_moment_decomp}Note that we cannot
use \cref{eq:moments_diffraction_real} here because this equation was
derived using the symmetry property of the physical moments in
\cref{eq:diffraction_moment_sym_2}, which we derived from the symmetry
properties of the Clebsch-Gordan coefficients.  However, since
$\eta(\Omega)$ can be an arbitrary function, the measured moments do
in general not have this symmetry property.  Note that
\cref{eq:diffraction_moment_sym_1} still holds, because it is related
to the symmetry properties of the spherical harmonics in
\cref{eq:spherical_harm_sym}.  This means that also for the measured
moments we have to calculate only those with $M \geq 0$.
Alternatively, one can use the real-valued spherical harmonics $Y_{L
M}$ (note the different position of the $M$ index) as defined, for
example, in \refsCite{wikipedia:sphericalHarm,Jones:2023} with $-L
\leq M \leq +L$.} (\confer \cref{eq:moments_diffraction_norm};
equivalent to Eq.~(B2) in \refCite{E852:1999xev})
\begin{equation}
  \label{eq:moments_diffraction_meas}
  H_\text{meas}(L, M)
  = \sqrt{\frac{4 \pi}{2 L + 1}}\, \int_{4 \pi}\!\!\! \dif{\Omega}\, \mathcal{I}_\text{meas}(\Omega)\, Y_L^{M *}(\Omega)
  = \sqrt{\frac{4 \pi}{2 L + 1}}\, \int_{4 \pi}\!\!\! \dif{\Omega}\, \eta(\Omega)\, \mathcal{I}(\Omega)\, Y_L^{M *}(\Omega).
\end{equation}

Inserting the moment decomposition for the physical intensity distribution
from \cref{eq:intensity_moments_diffraction_real} into
\cref{eq:moments_diffraction_meas} we get
\begin{align}
  H_\text{meas}(L, M)
  ={}& \sqrt{\frac{4 \pi}{2 L + 1}}\, \int_{4 \pi}\!\!\! \dif{\Omega}\, \eta(\Omega)\,
  \sum_{L' = 0}^\infty \sqrt{\frac{2 L' + 1}{4 \pi}} \sum_{M' = 0}^{L'} H(L', M')\, (2 - \delta_{M' 0})\, y_{L'}^{M'}(\theta)\, \cos(M'\, \phi)\,
  Y_L^{M *}(\Omega) \notag \\
  \label{eq:moments_diffraction_meas2}
  ={}& \sum_{L' = 0}^\infty \sum_{M' = 0}^{L'} H(L', M')
  \dUnderbrace{\sqrt{\frac{2 L' + 1}{2 L + 1}}\, (2 - \delta_{M' 0}) \int_{4 \pi}\!\!\! \dif{\Omega}\, \eta(\Omega)\,
  y_{L'}^{M'}(\theta)\, \cos(M'\, \phi)\, Y_L^{M *}(\Omega)}{\equiv I^\text{acc}_{L M\, L' M'}},
\end{align}
which is equivalent to Eqs.~(B3) and~(B4) in \refCite{E852:1999xev}.
Here, $I^\text{acc}_{L M\, L' M'}$ are the overlap integrals of the
spherical harmonics in the accepted phase space.

In practice, we want to calculate only a finite set of moments and
only a finite set of $(L' M')$ quantum numbers contribute to each of
these moments.  We can hence rewrite
\cref{eq:moments_diffraction_meas2} as a matrix equation in
quantum-number space, \ie
\begin{equation}
  \mathbf{H}_\text{meas}
  = \mathbf{I}^\text{acc}\, \mathbf{H},
\end{equation}
where $\mathbf{I}^\text{acc}$ is the acceptance integral matrix.  For
an ideal detector with $\eta(\Omega) = 1$, $\mathbf{I}^\text{acc}$
becomes a unit matrix in quantum-number space, \ie
\begin{equation}
  \label{eq:diffraction_integral_matrix_ideal}
  \rBrk{\mathbf{I}^\text{acc}}_{L M\, L' M'}
  = I^\text{acc}_{L M\, L' M'}
  = \delta_{L L'}\, \delta_{M M'},
\end{equation}
because of the orthonormality of the spherical harmonics (see
\cref{eq:spherical_harm_orthonorm}).\footnote{%
  For $\eta(\Omega) = 1$, the overlap integrals are
  \begin{equation}
    I^\text{acc}_{L M\, L' M'}
    = \sqrt{\frac{2 L' + 1}{2 L + 1}}\, (2 - \delta_{M' 0})
    \int_{-1}^{+1}\!\!\! \dif{\cos\theta}\, y_{L'}^{M'}(\theta)\, y_L^M(\theta)
    \int_{-\pi}^{+\pi}\!\!\! \dif{\phi}\, \cos(M'\, \phi)\, \sBrk[2]{\cos(M\, \phi) + \imag\, \sin(M\, \phi)}.
  \end{equation}
  The $\sin$ and $\cos$ functions are orthogonal, \ie
  \begin{equation}
    \int_{-\pi}^{+\pi}\!\!\! \dif{\phi}\, \cos(M'\, \phi)\, \sin(M\, \phi)
    = 0
    \quad\text{and}\quad
    \int_{-\pi}^{+\pi}\!\!\! \dif{\phi}\, \cos(M'\, \phi)\, \cos(M\, \phi)
    = \begin{cases*}
      0    & for $M \neq M'$ \\
      \pi  & for $M = M' \neq 0$ \\
      2\pi & for $M = M' = 0$.
    \end{cases*}
  \end{equation}
  This means that
  \begin{equation}
    (2 - \delta_{M' 0})
    \int_{-\pi}^{+\pi}\!\!\! \dif{\phi}\, \cos(M'\, \phi)\, \sBrk[2]{\cos(M\, \phi) + \imag\, \sin(M\, \phi)}
    = 2 \pi\, \delta_{M M'}.
  \end{equation}
  Using this and the definition of the spherical harmonics in
  \cref{eq:sspherical_harm_def}, we get
  \begin{equation}
    I^\text{acc}_{L M\, L' M'}
    = \sqrt{\frac{2 L' + 1}{2 L + 1}}\, 2 \pi\, \delta_{M M'}\,
    \sqrt{\frac{2 L' + 1}{4 \pi}\, \frac{(L' - M)!}{(L' + M)!}}\,
    \sqrt{\frac{2 L + 1}{4 \pi}\, \frac{(L - M)!}{(L + M)!}}\,
    \int_{-1}^{+1}\!\!\! \dif{\cos\theta}\, P_{L'}^M(\cos\theta)\, P_L^M(\cos\theta)
  \end{equation}
  Inserting the orthogonality of the associated Legendre polynomials
  (see \eg\ \refCite{wikipedia:accocLegendrePol}), \ie
  \begin{equation}
    \int_{-1}^{+1}\!\!\! \dif{\cos\theta}\, P_{L'}^M(\cos\theta)\, P_L^M(\cos\theta)
    = \frac{2}{2 L + 1}\, \frac{(L + M)!}{(L - M)!}\, \delta_{L L'},
  \end{equation}
  we obtain \cref{eq:diffraction_integral_matrix_ideal}.
}
In this case, $\mathbf{H}_\text{meas} = \mathbf{H}$, as expected.

For a realistic detector, $\eta(\Omega) \neq 1$ will break the
orthogonality of the spherical harmonics.  In this case, the diagonal
elements of $\mathbf{I}^\text{acc}$ represent the acceptance for the
intensity distribution that corresponds to the respective $(L M)$
quantum numbers.  The off-diagonal elements of $\mathbf{I}^\text{acc}$
are in general non-zero and are related to the indistinguishability of
moments with $(L M)$ from those with $(L' M')$ quantum numbers due to
the limited detection efficiency.  This indistinguishability between
quantum-number combinations can be quantified by considering the
normalized acceptance integral matrix
\begin{equation}
  \tilde{\mathbf{I}}^\text{acc}
  = \diag(\mathbf{I}^\text{acc})^{-1/2}\, \mathbf{I}^\text{acc}\, \diag(\mathbf{I}^\text{acc})^{-1/2},
  \quad\text{\ie}\quad
  \tilde{I}^\text{acc}_{L M\, L' M'}
  = \frac{I^\text{acc}_{L M\, L' M'}}{\sqrt{I^\text{acc}_{L M\, L M}}\, \sqrt{I^\text{acc}_{L' M'\, L' M'}}}.
\end{equation}
If the angular distributions for two moments become indistinguishable
in the accepted phase space, the corresponding off-diagonal elements
of $\tilde{\mathbf{I}}^\text{acc}$ will approach~1.

$\mathbf{I}^\text{acc}$ can be calculated using Monte Carlo
integration.  To this end, we first generate $N_\text{gen}$~events
that are uniformly distributed in the two-body phase
space\footnote{This means the events are uniformly distributed in the
$(\cos\theta, \phi)$ plane.} and then pass these events through the
detector simulation, event reconstruction, and event selection chains.
From the remaining $N_\text{acc}$ accepted events we calculate the
elements of the acceptance integral matrix using
\begin{equation}
  \label{eq:diffraction_integral_matrix}
  \rBrk{\mathbf{I}^\text{acc}}_{L M\, L' M'}
  = I^\text{acc}_{L M\, L' M'}
  = \sqrt{\frac{2 L' + 1}{2 L + 1}}\, (2 - \delta_{M' 0})\,
  \frac{4 \pi}{N_\text{gen}} \sum_{i = 1}^{N_\text{acc}} y_{L'}^{M'}(\theta_i)\, \cos(M'\, \phi_i)\, Y_L^{M *}(\Omega_i).
\end{equation}
Expect for the factor $4 \pi$ that represents the integration volume,
this is identical to Eq.~(B6) in \refCite{E852:1999xev}.

If the integral matrix $\mathbf{I}^\text{acc}$ is invertible, we can
calculate the physical moments from the measured ones using
\begin{equation}
  \label{eq:diffraction_phys_moments}
  \mathbf{H}
  = \rBrk[1]{\mathbf{I}^\text{acc}}^{-1}\, \mathbf{H}_\text{meas}.
\end{equation}
In case the detection efficiency is close to zero over wide ranges of
the phase space, $\mathbf{I}^\text{acc}$ may become (nearly) singular.
In this case, certain moments may become immeasurable and one has to
revert to the support-vector moments developed by
R.~Jones~\cite{Jones:2023}.


\subsubsection{Estimation of measured moments and uncertainty propagation}%
\label{sec:diffraction:estimation_uncert}

To obtain an estimate for $\mathbf{H}_\text{meas}$ from
\cref{eq:moments_diffraction_meas} we use the same approach as in
\cref{eq:moments_diffraction_estimate}, \ie
\begin{equation}
  \label{eq:moments_meas_diffraction_estimate}
  \rBrk[1]{\hat{\mathbf{H}}_\text{meas}}_{L, M}
  = \hat{H}_\text{meas}(L, M)
  = \sum_{i = 1}^{N_\text{meas}} \sqrt{\frac{4 \pi}{2 L + 1}}\, Y_L^{M *}(\Omega_i)
  \equiv \sum_{i = 1}^{N_\text{meas}} f_{L M}(\Omega_i).
\end{equation}
This means, the measured moments are given by the
spherical-harmonics-weighted sum of the $N_\text{meas}$~measured
events.  From the estimates of the measured moments we calculate the
estimates~$\hat{\mathbf{H}}$ of the physical moments using
\cref{eq:diffraction_phys_moments}.

We also have to propagate the uncertainty
from~$\hat{\mathbf{H}}_\text{meas}$ to~$\hat{\mathbf{H}}$.  This is
complicated by the fact that in general the measured moments
$\hat{\mathbf{H}}_\text{meas}$~as well as the inverse of the
acceptance matrix $\rBrk[1]{\mathbf{I}^\text{acc}}^{-1}$ in
\cref{eq:diffraction_phys_moments} are complex-valued.  The most
convenient way to formulate uncertainty propagation for complex-valued
quantities is to use Wirtinger calculus\footnote{Sometimes this is
also referred to as
$\mathbb{C}\mathbb{R}$-calculus.}~\cite{wikipedia:WirtingerCalculus,Wirtinger:1927,Kreutz-Delgado:2009,Grube:2023}.
In this approach, the linear uncertainty propagation for
\cref{eq:diffraction_phys_moments} takes on the familiar
form\footnote{We use~$\dagger$ to indicate the Hermitian conjugate.}
\begin{equation}
  \label{eq:complex_uncert_prop}
  \underaccent{\bar}{\covMatSym}_{\!H}
  = \underaccent{\bar}{\mathbf{J}}\, \underaccent{\bar}{\covMatSym}_{\!H_\text{meas}}\, \underaccent{\bar}{\mathbf{J}}^\dagger.
\end{equation}
Here, the matrices are expressed in the so-called augmented
representation (see \eg\ \refsCite{Adali:2014,Grube:2023}), which is
indicated by the underbars.  This representation is obtained by
stacking the moment vector~$\mathbf{H}$ on top of its
complex-conjugate vector~$\mathbf{H}^*$, \ie the augmented moment
vector is
\begin{equation}
  \underaccent{\bar}{\mathbf{H}}
  \equiv \begin{pmatrix*}[l]
    \mathbf{H} \\
    \mathbf{H}^*
  \end{pmatrix*}
  \in \mathbb{C}^{2n}
\end{equation}
with $n$~being the number of moments.

In the augmented representation, the covariance matrix
of~$\underaccent{\bar}{\mathbf{H}}$\footnote{Analogous expressions
define the covariance matrix
of~$\underaccent{\bar}{\mathbf{H}}_\text{meas}$.} is a $\mathbb{C}^{2n
\times 2n}$ matrix, \ie
\begin{equation}
  \underaccent{\bar}{\covMatSym}_{\!H}
  = \cov[1]{\underaccent{\bar}{\mathbf{H}}, \underaccent{\bar}{\mathbf{H}}}
  = \exptVal[2]{\rBrk[1]{\underaccent{\bar}{\mathbf{H}} - \exptVal[1]{\underaccent{\bar}{\mathbf{H}}}}\,
  \rBrk[1]{\underaccent{\bar}{\mathbf{H}} - \exptVal[1]{\underaccent{\bar}{\mathbf{H}}}}^\dagger}.
\end{equation}
This matrix has a block structure of the form
\begin{equation}
  \label{eq:cov_augmented}
  \underaccent{\bar}{\covMatSym}_{\!H}
  = \begin{pmatrix*}[l]
    \covMatSym_{\!H} & \tilde{\covMatSym}_{\!H} \\
    \tilde{\covMatSym}_{\!H}^* & \covMatSym_{\!H}^*
  \end{pmatrix*}
\end{equation}
where
\begin{equation}
  \covMatSym_{\!H}
  = \cov[1]{\mathbf{H}, \mathbf{H}}
  = \exptVal[2]{\rBrk[1]{\mathbf{H} - \exptVal[1]{\mathbf{H}}}\, \rBrk[1]{\mathbf{H} - \exptVal[1]{\mathbf{H}}}^\dagger}
\end{equation}
is the $\mathbb{C}^{n \times n}$ Hermitian covariance matrix
of~$\mathbf{H}$ and
\begin{equation}
  \tilde{\covMatSym}_{\!H}
  = \cov[1]{\mathbf{H}, \mathbf{H}^*}
  = \exptVal[2]{\rBrk[1]{\mathbf{H} - \exptVal[1]{\mathbf{H}}}\, \rBrk[1]{\mathbf{H} - \exptVal[1]{\mathbf{H}}}^T}
\end{equation}
is the $\mathbb{C}^{n \times n}$ pseudo-covariance matrix
of~$\mathbf{H}$.  These matrices are related to the covariance
matrices of the real and imaginary parts of~$\mathbf{H}$ in the
following way:
\begin{align}
  \label{eq:cov_ReRe}
  \covMatSym_{\!H}^{RR}
  = \exptVal[2]{\rBrk[1]{\Re\!\sBrk{\mathbf{H}} - \exptVal[1]{\Re\!\sBrk{\mathbf{H}}}}\, \rBrk[1]{\Re\!\sBrk{\mathbf{H}} - \exptVal[1]{\Re\!\sBrk{\mathbf{H}}}}^T}
  ={}& \frac{1}{2}\, \Re\!\sBrk[1]{\covMatSym_{\!H} + \tilde{\covMatSym}_{\!H}}
  \\
  \label{eq:cov_ImIm}
  \covMatSym_{\!H}^{II}
  = \exptVal[2]{\rBrk[1]{\Im\!\sBrk{\mathbf{H}} - \exptVal[1]{\Im\!\sBrk{\mathbf{H}}}}\, \rBrk[1]{\Im\!\sBrk{\mathbf{H}} - \exptVal[1]{\Im\!\sBrk{\mathbf{H}}}}^T}
  ={}& \frac{1}{2}\, \Re\!\sBrk[1]{\covMatSym_{\!H} - \tilde{\covMatSym}_{\!H}}
  \\
  \label{eq:cov_ReIm}
  \covMatSym_{\!H}^{RI}
  = \exptVal[2]{\rBrk[1]{\Re\!\sBrk{\mathbf{H}} - \exptVal[1]{\Re\!\sBrk{\mathbf{H}}}}\, \rBrk[1]{\Im\!\sBrk{\mathbf{H}} - \exptVal[1]{\Im\!\sBrk{\mathbf{H}}}}^T}
  ={}& \frac{1}{2}\, \Im\!\sBrk[1]{\covMatSym_{\!H} - \tilde{\covMatSym}_{\!H}}
  = \rBrk[1]{\covMatSym_{\!H}^{IR}}^T.
\end{align}
And conversely,
\begin{align}
  \covMatSym_{\!H}
  = \covMatSym_{\!H}^{RR} + \covMatSym_{\!H}^{II} + \imag \rBrk{\covMatSym_{\!H}^{IR} - \covMatSym_{\!H}^{RI}}
  \\
  \tilde{\covMatSym}_{\!H}
  = \covMatSym_{\!H}^{RR} - \covMatSym_{\!H}^{II} + \imag \rBrk{\covMatSym_{\!H}^{IR} + \covMatSym_{\!H}^{RI}}
\end{align}

Analogous to \cref{eq:diffraction_sample_cov} we can estimate the
Hermitian and the pseudo-covariance matrix for the measured moments
using the sample covariance:\footnote{In NumPy,
$\hat{\covMatSym}_{\!\hat{H}_\text{meas}}$~can be calculated directly
by calling \pyinline{numpy.cov(f)}, where \pyinline{f} is an array of
$N_\text{meas}$ vectors with the values of $f_{L M}(\Omega_i)$ as
defined in \cref{eq:moments_meas_diffraction_estimate}.  However, to
calculate the pseudo-covariance matrix
$\hat{\tilde{\covMatSym}}_{\!\hat{H}_\text{meas}}$~in NumPy we have to
use \pyinline{numpy.cov(f, numpy.conjugate(f))[:n, n:]}.}
\begin{align}
  (\hat{\covMatSym}_{\!\hat{H}_\text{meas}})_{L M, L' M'}
  ={}& \cov[2]{\hat{H}_\text{meas}(L, M), \hat{H}_\text{meas}(L', M')} \notag
  \\
  \label{eq:diffraction_sample_cov_hermit_meas}
  ={}& \frac{N_\text{meas}}{N_\text{meas} - 1}\, \sum_{i = 1}^{N_\text{meas}}
  \sBrk[2]{f_{L M}(\Omega_i) - \mean{f_{L M}(\Omega)}}\, \sBrk[2]{f_{L' M'}(\Omega_i) - \mean{f_{L' M'}(\Omega)}}^*.
  \\
  (\hat{\tilde{\covMatSym}}_{\!\hat{H}_\text{meas}})_{L M, L' M'}
  ={}& \cov[2]{\hat{H}_\text{meas}(L, M), \hat{H}_\text{meas}^*(L', M')} \notag
  \\
  \label{eq:diffraction_sample_cov_pseudo_meas}
  ={}& \frac{N_\text{meas}}{N_\text{meas} - 1}\, \sum_{i = 1}^{N_\text{meas}}
  \sBrk[2]{f_{L M}(\Omega_i) - \mean{f_{L M}(\Omega)}}\, \sBrk[2]{f_{L' M'}(\Omega_i) - \mean{f_{L' M'}(\Omega)}}.
\end{align}
Here, $f_{L M}(\Omega)$ is defined in
\cref{eq:moments_meas_diffraction_estimate}.  From these two matrices,
we construct the augmented covariance
matrix~$\hat{\underaccent{\bar}{\covMatSym}}_{\!\hat{H}_\text{meas}}$
using \cref{eq:cov_augmented}.\footnote{In NumPy, we can
calculate~$\hat{\underaccent{\bar}{\covMatSym}}_{\!\hat{H}_\text{meas}}$
directly by calling \pyinline{numpy.cov(f, numpy.conjugate(f))}.
Alternatively, one can construct the augmented vector \pyinline{f_aug
= numpy.block([[f], [numpy.conjugate(f)]])} and call
\pyinline{numpy.cov(f_aug)}.}

In \cref{eq:complex_uncert_prop}, $\underaccent{\bar}{\mathbf{J}}$ is
the $\mathbb{C}^{2n \times 2n}$ augmented Jacobian matrix of
\cref{eq:diffraction_phys_moments}, which has a similar block
structure as the augmented covariance matrix in
\cref{eq:cov_augmented}:
\begin{equation}
  \label{eq:J_augmented}
  \underaccent{\bar}{\mathbf{J}}
  = \begin{pmatrix*}[l]
    \mathbf{J} & \tilde{\mathbf{J}} \\
    \tilde{\mathbf{J}}^* & \mathbf{J}^*
  \end{pmatrix*},
\end{equation}
where
\begin{equation}
  \mathbf{J}
  = \pd{\mathbf{H}}{\mathbf{H}_\text{meas}}
  = \begin{pmatrix}
    \dpd{H_1}{\mathbf{H}_\text{meas}} \\
    \vdots \\
    \dpd{H_n}{\mathbf{H}_\text{meas}}
  \end{pmatrix}
  = \begin{pmatrix}
    \dpd{H_1}{H_{\text{meas}, 1}} & \ldots & \dpd{H_1}{H_{\text{meas}, n}} \\
    \vdots & & \vdots \\
    \dpd{H_n}{H_{\text{meas}, 1}} & \ldots & \dpd{H_n}{H_{\text{meas}, n}}
  \end{pmatrix}
\end{equation}
is the $\mathbb{C}^{n \times n}$ Jacobian matrix \wrt~$\mathbf{H}$ and
\begin{equation}
  \tilde{\mathbf{J}}
  = \pd{\mathbf{H}}{{\mathbf{H}_\text{meas}^*}}
\end{equation}
is the $\mathbb{C}^{n \times n}$ conjugate Jacobian matrix
\wrt~$\mathbf{H}^*$.

The advantage of Wirtinger calculus is that~$\mathbf{J}$
and~$\tilde{\mathbf{J}}$ are easily calculable using the same rules as
for real-valued expressions if one considers
\cref{eq:diffraction_phys_moments} as a function of two independent
variables, ~$\mathbf{H}$ and~$\mathbf{H}^*$.  Hence,
\begin{equation}
  \mathbf{J}
  = \pd{\mathbf{H}}{\mathbf{H}_\text{meas}}
  = \mathbf{A}
  \quad\text{and}\quad
  \tilde{\mathbf{J}}
  = \pd{\mathbf{H}}{{\mathbf{H}_\text{meas}^*}}
  = \mathbf{0},
\end{equation}
where we have applied the rules from matrix calculus (see \eg\
\refCite{wikipedia:MatrixCalculus}).  Consequently,
\begin{equation}
  \underaccent{\bar}{\mathbf{J}}
  = \begin{pmatrix*}[l]
    \mathbf{A} & \mathbf{0} \\
    \mathbf{0} & \mathbf{A}^*
  \end{pmatrix*}.
\end{equation}
With the above~$\underaccent{\bar}{\mathbf{J}}$ and
$\hat{\underaccent{\bar}{\covMatSym}}_{\!\hat{H}_\text{meas}}$~given
by
\cref{eq:diffraction_sample_cov_hermit_meas,eq:diffraction_sample_cov_pseudo_meas,eq:cov_augmented},
we can calculate the augmented covariance
matrix~$\hat{\underaccent{\bar}{\covMatSym}}_{\!\hat{H}}$ of the
physical moments using \cref{eq:complex_uncert_prop} and using
\cref{eq:cov_ReRe,eq:cov_ImIm,eq:cov_ReIm} the corresponding
covariance matrices of the real and imaginary parts of~$\hat{\mathbf{H}}$.

Finally, we can verify that the acceptance correction works correctly
by checking that the physical moments~$\hat{\mathbf{H}}$ fulfill
\cref{eq:moments_diffraction_imag}, \ie that they are real-valued
within the uncertainty.


\subsubsection{Moment decomposition of the detection efficiency}%
\label{sec:diffraction:acceptance_moment_decomp}

Following \refCite{E852:1999xev}, it is instructive to apply the
moment decomposition also to the detection efficiency $\eta(\Omega)$
in \cref{eq:diffraction_int_meas}.  Analogous to
\cref{eq:intensity_moments_diffraction_general} we obtain
\begin{align}
  \label{eq:diffraction_acc_moment_decomp}
  \eta(\Omega)
  ={}& \sum_{L = 0}^\infty \sqrt{4 \pi}\, \sqrt{2 L + 1} \sum_{M = -L}^{+L} H_\text{acc}(L, M)\, Y_L^M(\Omega)
  \\
  ={}& \sum_{L = 0}^\infty \sqrt{4 \pi}\, \sqrt{2 L + 1} \sum_{M = 0}^{L} (2 - \delta_{M 0})\, \Re\!\sBrk{H_\text{acc}(L, M)\, Y_L^M(\Omega)}
\end{align}
with the acceptance moments defined analogously to
\cref{eq:moments_diffraction_norm},\footnote{The same arguments as in
\cref{fn:complex_moment_decomp} apply here.} \ie
\begin{equation}
  \label{eq:diffraction_acc_moment}
  H_\text{acc}(L, M)
  = \frac{1}{\sqrt{4 \pi}\, \sqrt{2 L + 1}}\, \int_{4 \pi}\!\!\! \dif{\Omega}\, \eta(\Omega)\, Y_L^{M *}(\Omega).
\end{equation}
Note that we have used a slightly different normalization here, which
was chosen such that for perfect acceptance, \ie $\eta(\Omega) = 1$,
the only non-zero acceptance moment is $H_\text{acc}(0, 0) =
1$.\footnote{%
  This follows from (see Eq.~(1) in Sec.~5.9.1 of
  \refCite{Varshalovich:1988krb})
  \begin{equation}
    \int_{4 \pi}\!\!\! \dif{\Omega}\, Y_L^M(\Omega)
    = \sqrt{4 \pi}\, \delta_{L 0}\, \delta_{M 0}.
  \end{equation}
}

Inserting \cref{eq:diffraction_acc_moment_decomp} into the acceptance
integral matrix defined in \cref{eq:moments_diffraction_meas2}, we get
\begin{align}
  I^\text{acc}_{L M\, L' M'}
  ={}& \sqrt{\frac{2 L' + 1}{2 L + 1}} \sum_{L''\, M''}^\infty \sqrt{4 \pi}\, \sqrt{2 L'' + 1}\, H_\text{acc}(L'', M'')
  \int_{4 \pi}\!\!\! \dif{\Omega}\, Y_{L''}^{M''}(\Omega)\, Y_{L'}^{M'}(\Omega)\, Y_L^{M *}(\Omega) \notag
  \\
  \label{eq:diffraction_acc_int_moment_decomp}
  ={}& \frac{2 L' + 1}{2 L + 1} \sum_{L''\, M''}^\infty (2 L'' + 1)\,
  \clebsch{L''}{0}{L'}{0}{L}{0}\, \clebsch{L''}{M''}{L'}{M'}{L}{M}\, H_\text{acc}(L'', M'').
\end{align}
Here, we have switched back to using the complex-valued spherical
harmonics and including the negative $M$~values into the sums and we
have applied \cref{eq:spherical_harm_clebsch}.\todo{Why does Eq.~(B9)
in \refCite{E852:1999xev} contains $H_\text{acc}^*$?}  From the above
equation, it is clear that in \cref{eq:moments_diffraction_meas2} a
measured moment with given $(L, M)$ depends only on those physical
moments $H(L', M')$ and those acceptance moments $H_\text{acc}(L'',
M'')$, for which $L', L'' \leq 2 L$.  This means that under the
simplifying assumptions expressed in \cref{eq:diffraction_int_meas} we
need to know only a finite set of moments of the acceptance function.

\todoInl{discuss acceptance correction using moment decomposition of $\eta^{-1}(\Omega)$}

\todoInl{Add comment that one can check consistency of PWA and moment
analysis by comparing the moments directly calculated from data to the
ones calculated from the fitted partial-wave amplitudes.}
