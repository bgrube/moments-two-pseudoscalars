\section{Photoproduction with linearly polarized beam}%
\label{sec:photoprod}

\subsection{Moment decomposition}%
\label{sec:photoprod:moment}

Systems of two distinguishable spinless particles can also be produced
in photo-production reactions using stationary proton targets.  As for
the diffractive reactions, discussed in \cref{sec:diffraction}, we are
interested in the intermediate resonances~$X$ that are produced in
these reactions and that decay into the observed two-(pseudo)scalar
system.  A typical example for such a process is the reaction
\begin{equation}
  \gamma\, p \to X^0\, p \to \etaOrPr\pi^0\, p.
\end{equation}
How to analyze these reactions was studied in detail by the JPAC
collaboration in~\refsCite{Mathieu:2019fts,Mathieu:2019gxo}.  We will
follow these references here and will derive some of the
equations.\todo{Add figure with scattering process}

The analysis is based on the measured intensity distribution, \ie the
number density of events, which is decomposed into partial-wave
amplitudes:\footnote{This corresponds to Eq.~(A3) in
\refCite{Mathieu:2019fts} with the phase-space factor~$\kappa$
absorbed into the partial-wave amplitude, \ie with the redefinition
$\mathcal{T} \to \mathcal{T} / \sqrt{\kappa}$ (see also footnote~6 in
\refCite{Mathieu:2019fts}), and with Eq.~(A5) inserted.}
\begin{equation}
  \mathcal{I}(\Omega, \Phi; w, t)
  = \frac{\dif{N}}{\dif{w}\, \dif{t}\, \dif{\Omega}\, \dif{\Phi}}
  = \sum_{\substack{\ell m \\ \ell' m'}}^\infty
  Y_\ell^m(\Omega)~
  \dUnderbrace{\sum_{\substack{\lambda = \pm 1 \\ \mathclap{\lambda_1, \lambda_2 = \pm 1/2}}}
  \mathcal{T}^\ell_{\!m\, \lambda; \lambda_1\, \lambda_2}(w, t)\,
  \varrho^\gamma_{\lambda\, \lambda'}(\Phi)\,
  \mathcal{T}^{\ell' *}_{\!m'\, \lambda'; \lambda_1\, \lambda_2}(w, t)}{\equiv \varrho^{\ell\, \ell'}_{m\, m'}(w, t, \Phi)}~
  Y_{\ell'}^{m' *}(\Omega).
\end{equation}
Here, $N$~is the (acceptance-corrected) number of measured events,
$w$~is the invariant mass of the two-(pseudo)scalar system, $t$~is the
squared four-momentum transferred from the beam to the target
particle, $\Omega = (\theta, \phi)$ is the direction of one of the two
(pseudo)scalar mesons the~$X$ decays into, measured in the
Gottfried-Jackson or helicity rest frame of~$X$, and $\Phi$ is the
angle between the polarization vector of the linearly polarized photon
and the reaction plane in the $X$~rest frame.  The
$\mathcal{T}^\ell_{\!m\, \lambda; \lambda_1\, \lambda_2}(w, t)$ are
the partial-wave amplitudes that correspond to an intermediate state
that has a spin, which is given by the relative orbital angular
momentum~$\ell$ between the two-(pseudo)scalar mesons, and a spin
projection~$m$ \wrt the beam axis.  This intermediate state is
produced in the interaction of a photon with helicity~$\lambda$ and a
target proton with helicity~$\lambda_1$, where the recoil proton has
helicity~$\lambda_2$.  The angular distribution of the $X$~decay
products is given by the spherical harmonics
$Y_\ell^m(\Omega)$~\cite{wikipedia:sphericalHarm}; the dependence on
the polarization angle~$\Phi$ by the photon spin-density
matrix~$\mat{\upvarrho}^\gamma(\Phi)$.  In
\cref{eq:photoprod_intensity}, the $\varrho^{\ell\, \ell'}_{m\, m'}(w,
t, \Phi)$ are the elements of the spin-density of~$X$ analogous to the
diffractive case in \cref{eq:diffraction_intensity_refl_spin-dens}.

Using the fact that the three Pauli matrices $\vec{\mat{\upsigma}} =
(\mat{\upsigma}_1, \mat{\upsigma}_2, \mat{\upsigma}_3)^T$ together
with $\mat{I}_{2 \times 2}$ form a complete set in the space of
Hermitian $2 \times 2$ matrices, we can expand
$\mat{\upvarrho}^\gamma(\Phi)$, \ie
\begin{equation}
  \mat{\upvarrho}^\gamma(\Phi)
  = \frac{1}{2}\, \mat{I}_{2 \times 2} + \frac{1}{2}\, \vec{P}_\gamma(\Phi) \cdot \vec{\mat{\upsigma}}.
\end{equation}
Here, $\vec{P}_\gamma(\Phi)$ describes the photon polarization.  Its
length~$0 \leq P_\gamma \leq 1$ is the degree of polarization and its
direction depends on the kind of polarization:\footnote{See Eq.~(19)
in \refCite{Schilling:1969um}.}
\begin{equation}
  \vec{P}_\gamma(\Phi)
  = \begin{cases*}
    P_\gamma\, (0, 0, \lambda)^T                & for circularly polarized photons with $\lambda = \pm 1$, \\
    -P_\gamma\, (\cos 2 \Phi, \sin 2 \Phi, 0)^T & for linearly polarized photons with polarization angle~$\Phi$.
  \end{cases*}
\end{equation}
Hence, for linearly polarized photons, the intensity distribution in
\cref{eq:photoprod_intensity} can be written as the sum of three
terms:\footnote{See Eq.~(B4) in \refCite{Mathieu:2019fts}.}
\begin{equation}
  \mathcal{I}(\Omega, \Phi; w, t)
  = \mathcal{I}_0(\Omega; w, t)
  - \mathcal{I}_1(\Omega; w, t)\, P_\gamma\, \cos 2 \Phi
  - \mathcal{I}_2(\Omega; w, t)\, P_\gamma\, \sin 2 \Phi
\end{equation}
with
\begin{equation}
  \mathcal{I}_i(\Omega; w, t)
  = \sum_{\substack{\ell m \\ \ell' m'}}^\infty
  Y_\ell^m(\Omega)\,
  \prescript{i}{}{\varrho}^{\ell\, \ell'}_{m\, m'}(w, t, \Phi)\,
  Y_{\ell'}^{m' *}(\Omega)
\end{equation}
and\footnote{See Eq.~(11) in \refCite{Mathieu:2019fts}.}
\begin{align}
  \prescript{0}{}{\varrho}^{\ell\, \ell'}_{m\, m'}(w, t, \Phi)
  ={}& \frac{1}{2}\quad \sum_{\substack{\lambda = \pm 1 \\ \mathclap{\lambda_1, \lambda_2 = \pm 1/2}}}
  \mathcal{T}^\ell_{\!m\, \lambda; \lambda_1\, \lambda_2}(w, t)\,
  \mathcal{T}^{\ell' *}_{\!m'\, \lambda; \lambda_1\, \lambda_2}(w, t)
  = \prescript{0}{}{\varrho}^{\ell\, \ell'}_{m\, m'}(w, t)
  \\
  \prescript{1}{}{\varrho}^{\ell\, \ell'}_{m\, m'}(w, t, \Phi)
  ={}& \frac{1}{2}\quad \sum_{\substack{\lambda = \pm 1 \\ \mathclap{\lambda_1, \lambda_2 = \pm 1/2}}}
  \mathcal{T}^\ell_{\!m\, {-\lambda}; \lambda_1\, \lambda_2}(w, t)\,
  \mathcal{T}^{\ell' *}_{\!m'\, \lambda; \lambda_1\, \lambda_2}(w, t)
  \\
  \prescript{2}{}{\varrho}^{\ell\, \ell'}_{m\, m'}(w, t, \Phi)
  ={}& \frac{\imag}{2}\quad \sum_{\substack{\lambda = \pm 1 \\ \mathclap{\lambda_1, \lambda_2 = \pm 1/2}}}
  \lambda\,
  \mathcal{T}^\ell_{\!m\, {-\lambda}; \lambda_1\, \lambda_2}(w, t)\,
  \mathcal{T}^{\ell' *}_{\!m'\, \lambda; \lambda_1\, \lambda_2}(w, t).
\end{align}

